\documentclass[12pt,a4paper]{article}
\usepackage[UTF8]{ctex}                    % 中文支持
\usepackage{geometry}                      % 页面布局
\usepackage{amsmath,amssymb,amsthm}        % 数学符号与定理
\usepackage{mathtools,bm}                  % 高级数学工具
\usepackage{graphicx}                      % 图片插入
\usepackage{hyperref}                      % 超链接
\usepackage{booktabs}                      % 三线表
\usepackage[numbers,sort&compress]{natbib} % 参考文献
\usepackage{caption}                       % 图表标题
\usepackage[shortlabels]{enumitem}         % 列表环境

% ========== 页面布局 ==========
\geometry{left=2.5cm, right=2.5cm, top=2.5cm, bottom=2.5cm}
\setlength{\parskip}{0.5em}                % 段落间距
\renewcommand{\baselinestretch}{1.2}       % 行距

\begin{document}
	
	\section*{傅里叶变换学习大纲}
	
	\section{傅里叶变换的推导}
	
	\section{傅里叶变换的性质}
	\subsection{有界线性算子}
	\begin{itemize}
		\item \(L^1(\mathbb{R}) \to L^\infty(\mathbb{R})\)
		\item \(\|\mathcal{F}[f]\|_{L^\infty} \leq \frac{1}{\sqrt{2\pi}} \|f\|_{L^1}\)
	\end{itemize}
	
	\subsection{连续性}
	\begin{itemize}
		\item \(L^1(\mathbb{R}) \to C_b(\mathbb{R})\)
		\item \(\mathcal{F}[f]\) 在 \(\mathbb{R}\) 上连续
	\end{itemize}
	
	\subsection{三个经典性质}
	\begin{itemize}
		\item 平移
		\item 调制
		\item 伸缩
	\end{itemize}
	
	\subsection{微分性质}

		速降函数空间的应用

	
	\subsection{积分性质}



	
	\subsection{乘积公式}

	

	
	\section{卷积与卷积傅里叶变换的性质}
	\begin{itemize}
		\item 定义:\(f * g(x) = \int_{\mathbb{R}} f(y)g(x - y) dy\)
		\item 性质:\(\mathcal{F}[f * g] = \mathcal{F}[f] \cdot \mathcal{F}[g]\)
	\end{itemize}
	
	\section{傅里叶逆变换的性质}
	\begin{itemize}
		\item 定义:\(\mathcal{F}^{-1}[g](x) = \int_{\mathbb{R}} g(\omega) e^{i\omega x} d\omega\)
		\item 性质:\(\mathcal{F}^{-1}[\mathcal{F}[f]] = f\) 和 \(\mathcal{F}[\mathcal{F}^{-1}[g]] = g\)
	\end{itemize}
	
	\section{傅里叶变换从 \(L^1(\mathbb{R}) \cap L^2(\mathbb{R})\) 到 \(L^2(\mathbb{R})\)}
	\begin{itemize}
		\item 映射原因:完备性和保能量性质
		\item 性质:等距满射,酉算子
		\item Parseval 定理:\(\|\mathcal{F}[f]\|_{L^2} = \|f\|_{L^2}\)
	\end{itemize}
	
	\section{傅里叶变换从 \(\mathcal{S}(\mathbb{R})\) 到 \(\mathcal{S}(\mathbb{R})\)}
	\begin{itemize}
		\item 速降函数空间的定义
		\item \(\mathcal{S}(\mathbb{R}) \subset L^2(\mathbb{R})\)
		\item \(\mathcal{S}(\mathbb{R})\) 在 \(L^2(\mathbb{R})\) 中的稠密性
		\item \(C_c^\infty(\mathbb{R}) \subset \mathcal{S}(\mathbb{R})\)
		\item 经典例子:高斯函数
	\end{itemize}
	
	\section{广义函数的傅里叶变换}
	\begin{itemize}
		\item 广义函数定义
		\item 例子:狄利克雷函数
		\item \(\mathcal{F}[\delta](\omega) = \frac{1}{\sqrt{2\pi}}\)
	\end{itemize}
	
	\section{离散傅里叶变换及其改进快速傅立叶变换}
	\begin{itemize}
		\item 离散傅里叶变换:\(X_k = \sum_{n=0}^{N-1} x_n e^{-i2\pi kn/N}\)
		\item 快速傅立叶变换:计算复杂度 \(O(N \log N)\)
	\end{itemize}
	
\end{document}