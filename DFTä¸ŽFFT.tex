\documentclass[12pt,a4paper]{article}
\usepackage[UTF8]{ctex}
\usepackage{geometry}
\usepackage{amsmath,amssymb,amsthm}
\usepackage{mathtools,bm}
\usepackage{empheq}
\usepackage{graphicx}
\usepackage{booktabs}
\usepackage[numbers,sort&compress]{natbib}
\usepackage{caption}
\usepackage{enumitem}
\usepackage{chngcntr}

% ========== 页面布局 ==========
\geometry{left=2.5cm,right=2.5cm,top=2.5cm,bottom=2.5cm}
\setlength{\parskip}{0.5em}
\renewcommand{\baselinestretch}{1.2}

% ========== 数学命令 ==========
\newcommand{\diff}{\mathop{}\!\mathrm{d}}
\newcommand{\R}{\mathbb{R}}
\newcommand{\C}{\mathbb{C}}
\newcommand{\Z}{\mathbb{Z}}
\newcommand{\N}{\mathbb{N}}
\DeclareMathOperator{\supp}{supp}

% ========== 编号系统 ==========
\numberwithin{subsection}{section}   % 子小节按章节编号
\numberwithin{subsubsection}{subsection}
\counterwithin{equation}{subsection} % 公式按子小节编号

% ========== 定理环境 ==========
\theoremstyle{plain}
\newtheorem{theorem}{定理}[section]
\newtheorem{lemma}[theorem]{引理}
\newtheorem{proposition}[theorem]{命题}
\newtheorem{corollary}[theorem]{推论}
\newtheorem{solution}{解}[subsection]  % 解按子小节编号

\theoremstyle{definition}
\newtheorem{definition}[theorem]{定义}
\newtheorem{example}{例题}[subsection]  % 示例按子小节编号

\theoremstyle{remark}
\newtheorem{remark}{注记}[subsection] % 直接在定义时指定计数器的层级关系

\theoremstyle{remark}
\newtheorem{verification}[theorem]{验证}

% 加载hyperref包以实现超链接功能
\usepackage[colorlinks=true, linkcolor=black]{hyperref}

\title{DFT 与 FFT}
\author{陈柏均}
\date{2025年4月25日}

\begin{document}
	\maketitle
	
	\section{DFT}
	
	\subsection{信号}
	 $\{x_k: k=0,\dots,N-1\}$,记 $\omega_N = e^{+2\pi i / N}$,其中$z^n=1$
	
	\begin{equation}
		y_k = \frac{1}{N} \sum_{n=0}^{N-1} x_n e^{-2\pi i \frac{nk}{N}} = \frac{1}{N} \sum_{n=0}^{N-1} x_n \omega_N^{-nk}
	\end{equation}
	
	称 $\{y_0,\dots,y_{N-1}\}$ 为 $\{x_0,\dots,x_{N-1}\}$ 的 DFT。
	
	记 $\mathcal{F}_N: \C^N \rightarrow \C^N,\quad \mathcal{F}_N(x) = y$。
	

		\subsection{ 逆公式}
		
		\begin{equation}
			x_k = \sum_{n=0}^{N-1} y_n e^{2\pi i \frac{nk}{N}} = \sum_{n=0}^{N-1} y_n \omega_N^{nk}, \text{即} \mathcal{F}_N^{-1}(y) = x.
		\end{equation}
		
		\begin{proof}
	\begin{equation}
		\begin{aligned}
			\sum_{n=0}^{N-1} y_n \omega_N^{nk} 
			&= \sum_{n=0}^{N-1} \left( \frac{1}{N} \sum_{r=0}^{N-1} x_r \omega_N^{-rn} \right) \omega_N^{nk} \\[8pt]
			&= \sum_{r=0}^{N-1} x_r \frac{1}{N} \sum_{n=0}^{N-1} \omega_N^{n(k - r)} \\[8pt]
			&= \sum_{r=0}^{N-1} x_r \delta_{k,r} = x_k.
		\end{aligned}
	\end{equation}
		其中
			\begin{equation}
				\delta_{k,r} = 
				\begin{cases} 
					1, & k = r \\[8pt]
					0, & k \neq r 
				\end{cases}
			\end{equation}
			
\end{proof}

\begin{equation}
\frac{1}{N}\sum_{n=0}^{N-1}\omega_N^{n(k-r)}=\delta_{k,r}
\end{equation}

	\begin{proof}
		当 $k=r$ 时,显然成立。
		
		当 $k \neq r$ 时,
		
\begin{equation}
	\begin{aligned}
		\frac{1}{N}\sum_{n=0}^{N-1}\omega_N^{n(k-r)} &= \frac{1}{N} \cdot \frac{1 - \omega_N^{N(k-r)}}{1 - \omega_N^{k-r}} \cdot \omega_N^{k-r} \\[8pt]
		&= \frac{1}{N} \cdot \frac{1 - \omega_N^{N(k-r)}}{1 - \omega_N^{k-r}} \\[8pt]
		&= \frac{1}{N} \cdot \frac{1 - (\omega_N^N)^{k-r}}{1 - \omega_N^{k-r}} = 0
	\end{aligned}
\end{equation}
这里用到了 $\omega_N^N = 1$。
	\end{proof}
	
	\subsection{Fourier矩阵}
	
	\begin{equation}
		\begin{bmatrix}
			y_0 \\
			y_1 \\
			\vdots \\
			y_{N-1}
		\end{bmatrix}
		= F_N
		\begin{bmatrix}
			x_0 \\
			x_1 \\
			\vdots \\
			x_{N-1}
		\end{bmatrix},
	\end{equation}

	\begin{equation}
		F_N = \left( \omega_N^{-nk} \right)_{n,k=0}^{N-1}
		=
		\begin{bmatrix}
			1 & 1 & 1 & \cdots & 1 \\
			1 & \omega_N & \omega_N^2 & \cdots & \omega_N^{N-1} \\
			1 & \omega_N^2 & \omega_N^4 & \cdots & \omega_N^{2(N-1)} \\
			\vdots & \vdots & \vdots & \ddots & \vdots \\
			1 & \omega_N^{N-1} & \omega_N^{2(N-1)} & \cdots & \omega_N^{(N-1)^2}
		\end{bmatrix}
	\end{equation}
	
\subsubsection{性质1}
		 $F_N$是酉矩阵,$F_N^* F_N = N I_{N \times N}$
		

考虑第$j$行$(\omega_N^{j \times 0}, \omega_N^{j \times 1}, \omega_N^{j \times 2}, \ldots, \omega_N^{j(N-1)}) \equiv J$
			
			对于第$k$行$(\omega_N^{k \times 0}, \omega_N^{k \times 1}, \omega_N^{k \times 2}, \ldots, \omega_N^{k(N-1)}) \equiv K$
			
			
			\begin{proof}不同行正交
	\begin{equation}
		\begin{aligned}
			J^T \overline{K} &= \omega_N^{j \times 0} \overline{\omega_N^{k \times 0}} + \omega_N^{j \times 1} \overline{\omega_N^{k \times 1}} + \cdots + \omega_N^{j(N-1)} \overline{\omega_N^{k(N-1)}} \\[8pt]
			&= \omega_N^{(j - k) \times 0} + \omega_N^{(j - k) \times 1} + \cdots + \omega_N^{(j - k)(N-1)} \\[8pt]
			&= \sum_{m=0}^{N-1} \omega_N^{(j - k)m} = \frac{1 - \omega_N^{(j - k)N}}{1 - \omega_N^{j - k}} = 0
		\end{aligned}
	\end{equation}
			这里用到了$\omega_N^N = 1$。
			\end{proof}
			
				\begin{proof}不同列正交
	           
	           由上同理可得
				\end{proof}
			
	 \begin{proof}同一行(列)的内积为$N$。
			
			\begin{equation}
				\begin{aligned}
					J^T \overline{J} &= \omega_N^{j \times 0} \overline{\omega_N^{j \times 0}} + \omega_N^{j \times 1} \overline{\omega_N^{j \times 1}} + \cdots + \omega_N^{j(N-1)} \overline{\omega_N^{j(N-1)}} \\[8pt]
					&= 1 + \cdots + 1 = N.
				\end{aligned}
			\end{equation}
		\end{proof}
	
		\subsubsection{性质2}
	\begin{equation}
		F_N^{-1} = \frac{1}{N} F_N^* = \left( \omega_N^{nk} \right)
	\end{equation}
	
	\subsection{ 二进制形式下的Fourier矩阵 ($N=2^p$)}
	
	\begin{equation}
		F_2 = \begin{pmatrix} 1 & 1 \\ 1 & \omega_2^{-1 \times 1} \end{pmatrix}
		= \begin{pmatrix} 1 & 1 \\ 1 & e^{-i \frac{2\pi}{2}} \end{pmatrix}
		= \begin{pmatrix} 1 & 1 \\ 1 & -1 \end{pmatrix}
	\end{equation}
	
	\begin{equation}
		\left\{
		\begin{aligned}
			y_0 &= \frac{1}{2}(x_0 + x_1 \omega_2^{-0 \times 1}) \\[8pt]
			y_1 &= \frac{1}{2}(x_0 + x_1 \omega_2^{-1 \times 1})
		\end{aligned}
		\right.
		\Leftrightarrow
		\begin{pmatrix} y_0 \\ y_1 \end{pmatrix}
		= \frac{1}{2} F_2 \begin{pmatrix} x_0 \\ x_1 \end{pmatrix}
		= \begin{pmatrix} \frac{1}{2}(x_0 + x_1) \\ \frac{1}{2}(x_0 - x_1) \end{pmatrix}
	\end{equation}
	
\begin{table}[h]
	\centering
	\caption{计算量}
	\begin{tabular}{|l|l|}
		\hline
		\textbf{total cost} & \textbf{矩阵计算} \\ \hline
		乘法次数 (multiplications): 0次 & 除法及计算: Add $2^1$ \\
		加法次数 (additions): 2次 & multi $2^2$ \\ \hline
	\end{tabular}
\end{table}
	
	\begin{equation}
		F_4 = \begin{pmatrix}
			1 & 1 & 1 & 1 \\
			1 & \omega_4^{-1 \times 1} & \omega_4^{-2 \times 1} & \omega_4^{-3 \times 1} \\
			1 & \omega_4^{-1 \times 2} & \omega_4^{-2 \times 2} & \omega_4^{-3 \times 2} \\
			1 & \omega_4^{-1 \times 3} & \omega_4^{-2 \times 3} & \omega_4^{-3 \times 3}
		\end{pmatrix}
		=
		\begin{pmatrix}
			1 & 1 & 1 & 1 \\
			1 & -i & -1 & i \\
			1 & -1 & 1 & -1 \\
			1 & i & -1 & -i
		\end{pmatrix}
	\end{equation}
	
	\begin{equation}
	\omega_4 = e^{\frac{2\pi i}{4}} = i,
		\end{equation}
	
	\subsection{ DFT的计算量 (cost of computation)}
	
	\begin{equation}
		y_k = \frac{1}{N} \sum_{n=0}^{N-1} x_n \omega_N^{-nk}
	\end{equation}
	
	每个$y_k$需要$(N-1)$个乘法,$N$个加法。
	
	总乘法次数:$(N-1)^2$
	
	总加法次数:$N(N-1)$
	
\begin{remark}
	计算$y_k$时不需要乘法,故不是$N^2$;用矩阵形式$y = F_N x$理解时,乘法次数为$N^2$;计算复杂度:$O(N^2)$!
\end{remark}

	
	\newpage
	
	\section{FFT}
	
	1965年,Cooley和Tukey提出FFT算法。(An algorithm for the machine calculation of complex Fourier series. Mathematics of Computation 19(90): 297-301, Jan 1965.
	DOI: 10.1090/S0025-5718-1965-0178586-1)	


\subsection{ FFT ($N=2^2=4$)}

\begin{equation}
	F_4: \{x_0, x_1, x_2, x_3\} \rightarrow \{y_0, y_1, y_2, y_3\}
\end{equation}

\begin{equation}
	y_k = \frac{1}{4} \left( x_0 \omega_4^{-0 \times k} + x_1 \omega_4^{-1 \times k} + x_2 \omega_4^{-2 \times k} + x_3 \omega_4^{-3 \times k} \right), \quad k = 0, 1, 2, 3
\end{equation}

奇偶重排 (rearrange) $y_k$:

\begin{equation}
	\begin{aligned}
		J^T \overline{K} &= \omega_N^{j \times 0} \overline{\omega_N^{k \times 0}} + \omega_N^{j \times 1} \overline{\omega_N^{k \times 1}} + \cdots + \omega_N^{j(N-1)} \overline{\omega_N^{k(N-1)}} \\[8pt]
		&= \omega_N^{(j - k) \times 0} + \omega_N^{(j - k) \times 1} + \cdots + \omega_N^{(j - k)(N-1)} \\[8pt]
		&= \sum_{m=0}^{N-1} \omega_N^{(j - k)m} = \frac{1 - \omega_N^{(j - k)N}}{1 - \omega_N^{j - k}} = 0
	\end{aligned}
\end{equation}


\begin{equation}
	P_k = \frac{x_0 + x_2 \omega_2^{-k}}{2}, \quad I_k = \frac{x_1 + x_3 \omega_2^{-k}}{2}, \quad k = 0, 1, 2, 3.
\end{equation}

显然:

\begin{equation}
	P_{k+2} = P_k, \quad k = 0, 1 \quad \text{和} \quad I_{k+2} = I_k, \quad k = 0, 1.
\end{equation}

因此:

\begin{equation}
	\begin{pmatrix} P_0 \\ P_1 \end{pmatrix} = F_2 \begin{pmatrix} x_0 \\ x_2 \end{pmatrix}, \quad \begin{pmatrix} I_0 \\ I_1 \end{pmatrix} = F_2 \begin{pmatrix} x_1 \\ x_3 \end{pmatrix}
\end{equation}

则:

\begin{equation}
	\left\{
	\begin{aligned}
		y_k &= \frac{1}{2} \left( P_k + \omega_4^{-k} I_k \right), \quad k = 0, 1 \\[8pt]
		y_{k+2} &= \frac{1}{2} \left( P_{k+2} + \omega_4^{-(k+2)} I_{k+2} \right) = \frac{1}{2} \left( P_k - \omega_4^{-k} I_k \right), \quad k = 0, 1
	\end{aligned}
	\right.
\end{equation}

至此,$\{ y_0, y_1, y_2, y_3 \}$ 全部算出。
	
\begin{table}[h]
	\centering
	\caption{计算量}
	\begin{tabular}{|c|c|c|c|c|}
		\hline
		& $P_k \ (k=0,1)$ & $I_k \ (k=0,1)$ & $y_k \ (k=0,1)$ & $y_{k+2} \ (k=0,1)$ \\ \hline
		乘法 & 0次 & 0次 & 1次 & 1次 \\ \hline
		加法 & 2次 & 2次 & 2次 & 2次 \\ \hline
	\end{tabular}
\end{table}

计算 $\{b_0, b_1, b_2, b_3\}$ 过程中的计算量:乘法 $1$ 次(表面上 $2$ 次),加法 $8$ 次

对比直接按公式计算的计算量:乘法 $(4-1)^2=9$,加法 $4 \times (4-1)=12$ 次
	
	\subsection{ FFT ($N=2^3=8$)}
	
	\begin{equation}
		F_8: \{x_0, x_1, x_2, x_3, x_4, x_5, x_6, x_7\} \rightarrow \{y_0, y_1, \ldots, y_7\}
	\end{equation}
	
\subsubsection{递推公式}
\begin{equation}
	\begin{aligned}
		y_j &= \frac{1}{8} \left( x_0 + x_1 \omega_8^{1 \times j} + x_2 \omega_8^{2 \times j} + \cdots + x_6 \omega_8^{6 \times j} + x_7 \omega_8^{7 \times j} \right) \\[8pt]
		&= \frac{1}{2} \left( \frac{1}{4} \left( x_0 + x_2 \omega_8^{2 \times j} + x_4 \omega_8^{4 \times j} + x_6 \omega_8^{-6 \times j} \right) \right. \\[8pt]
		&\quad \left. + \frac{1}{4} \left( x_1 \omega_8^{3 \times j} + x_3 \omega_8^{3 \times j} + x_5 \omega_8^{5 \times j} + x_7 \omega_8^{-7 \times j} \right) \right) \\[8pt]
		&= \frac{1}{2} \left( \frac{1}{4} \left( x_0 + x_2 \omega_4^{2 \times j} + x_4 \omega_4^{0 \times j} + x_6 \omega_4^{-0 \times j} \right. \right. \\[8pt]
		&\quad \left. \left. + x_1 \omega_4^{1 \times j} + x_3 \omega_4^{-1 \times j} + x_5 \omega_4^{2 \times j} + x_7 \omega_4^{-2 \times j} \right) \omega_8^{j} \right) \\[8pt]
		&= \frac{1}{2} \left( P_j^{(3)} + I_j^{(3)} \omega_8^{j} \right)
	\end{aligned}
\end{equation}

	
	其中:
	\begin{equation}
		P_j^{(3)} = \frac{1}{4} \left( x_0 + x_2 \omega_4^{2 \times j} + x_4 \omega_4^{0 \times j} + x_6 \omega_4^{-0 \times j} \right)
	\end{equation}
	
	\begin{equation}
		I_j^{(3)} = \frac{1}{4} \left( x_1 + x_3 \omega_4^{2 \times j} + x_5 \omega_4^{0 \times j} + x_7 \omega_4^{-0 \times j} \right)
	\end{equation}
	
可证:
	\begin{equation}
		\begin{cases}
			P_j^{(3)} = P_{j+4}^{(3)}, \quad j = 0, 1, 2, 3 \\[8pt]
			I_j^{(3)} = I_{j+4}^{(3)}
		\end{cases}
	\end{equation}
	
	\subsubsection{计算复杂度分析}
	
\textbf{第一层 (A)}
	

	$	\text{已知 } P_{j}^{(3)}, I_{j}^{(3)} (j=0,1,2,3), \text{ 计算 } P_{j}^{(3)} (j=0,1,2,3) \text{ 的计算量}$

	
	\begin{itemize}
		\item 乘法 3 次 (当 $j=0$ 时无乘法)
		\item 加法 8 次
	\end{itemize}
	
\textbf{第二层 (B)}
	

		$\text{下面讨论 } P_{j}^{(3)} \text{ 的计算量 (} j=0,1,2,3 \text{)}$

	
	一方面
	\begin{equation}
		\begin{aligned}
			P_{j}^{(3)} &= \frac{1}{4} \left( x_{0}, x_{2}, x_{4}, x_{6} \right) \cdot (1, w_{4}^{-j}, w_{4}^{-2j}, w_{4}^{-3j}) = \text{DFT} \left( x_{0}, x_{2}, x_{4}, x_{6} \right) \\[8pt]
			&= \frac{1}{2} \left\{ \frac{1}{2} \left( x_{0} + x_{4} w_{4}^{-2j} \right) + \frac{1}{2} \left( x_{2} + x_{6} w_{4}^{-2j} \right) w_{4}^{-j} \right\} \\[8pt]
			&= \frac{1}{2} \left\{ P_{j}^{(2)} + I_{j}^{(2)} w_{4}^{-j} \right\} \quad ( \text{同样地有 } I_{j}^{(3)} = P_{j}^{(2)} + I_{j}^{(2)} w_{4}^{-j} \quad j=0,1 )
		\end{aligned}
	\end{equation}
	
	如果已知 $P_{j}^{(2)}, I_{j}^{(2)}, j=0,1,2$,计算 $P_{j}^{(3)}$ 需要 1 次乘法,4 次加法。
	
	同样地
	\begin{equation}
		\begin{aligned}
			I_{j}^{(3)} &= \frac{1}{4} \left( x_{1}, x_{3}, x_{5}, x_{7} \right) \cdot (1, w_{4}^{-j}, w_{4}^{-2j}, w_{4}^{-3j}) \quad (j=0,1,2,3) \\[8pt]
			&= \frac{1}{2} \left\{ \frac{1}{2} \left( x_{1} + x_{5} w_{4}^{-2j} \right) + \frac{1}{2} \left( x_{3} + x_{7} w_{4}^{-2j} \right) w_{4}^{-j} \right\} \\[8pt]
			&= \frac{1}{2} \left\{ P_{j}^{(2)} + I_{j}^{(2)} w_{4}^{-j} \right\} \quad (j=0,1)
		\end{aligned}
	\end{equation}
	
	\textbf{三层 (C)}
	
	计算 $P_{j}^{(2)}, I_{j}^{(2)}, P_{j}^{(2)}, I_{j}^{(2)}$ ( $j=0,1$ ) 分别需要 0 次乘法,2 次加法。
	
 结论:(A)、(B)、(C) 三层共花费:乘法 3 + 2 + 0 = 5 \quad 加法 8 + 8 + 6 = 24
	
	\begin{remark}
	直接计算时,乘法 $4 \times 3 = 12$ 次,加法 $4 \times 3 = 12$ 次
	\end{remark}

\subsection{FFT ($N = 2^8$, general case)}

\subsubsection{定义和基本公式}

\begin{equation}
	F_{2^q}: \mathbb{C}^{2^q} \rightarrow \mathbb{C}^{2^q}
\end{equation}

\begin{equation}
	(x_0, x_1, \ldots, x_{2^q -1}) \rightarrow (y_0, y_1, \ldots, y_{2^q -1})
\end{equation}

\begin{equation}
	y_k = \frac{1}{2^q} (x_0, x_1, \ldots, x_{2^q -1}) \cdot \left(1, w_{2^q}^{-k}, w_{2^q}^{-2k}, \ldots, w_{2^q}^{-(2^q -1)k}\right)
\end{equation}

\begin{equation}
	y_k = \frac{1}{2^q} \sum_{j=0}^{2^q -1} x_j \left(w_{2^q}\right)^{-jk}, \quad k = 0, 1, \ldots, 2^q -1
\end{equation}

\subsubsection{递推公式}

\begin{equation}
	y_k = \frac{1}{2} \left\{ \frac{1}{2^{q-1}} \sum_{j=0}^{2^{q-1} -1} x_{2j} \left(w_{2^{q-1}}\right)^{-jk} + \frac{1}{2^{q-1}} \sum_{j=0}^{2^{q-1} -1} x_{2j+1} \left(w_{2^{q-1}}\right)^{-jk} \left(w_{2^q}\right)^{-k} \right\}
\end{equation}

\begin{equation}
	y_k = \frac{1}{2} \left( P_k^{(q-1)} + w_{2^q}^{-k} I_k^{(q-1)} \right), \quad k = 0, 1, \ldots, 2^{q} -1
\end{equation}

其中:
\begin{equation}
	P_k^{(q-1)} = \frac{1}{2^{q-1}} \sum_{j=0}^{2^{q-1} -1} x_{2j} \left(w_{2^{q-1}}\right)^{-jk}, \quad k = 0, 1, \ldots, 2^{q-1} -1
\end{equation}

\begin{equation}
	I_k^{(q-1)} = \frac{1}{2^{q-1}} \sum_{j=0}^{2^{q-1} -1} x_{2j+1} \left(w_{2^{q-1}}\right)^{-jk}, \quad k = 0, 1, \ldots, 2^{q-1} -1
\end{equation}

利用 $\left(w_{2^q}\right)^{-(k + 2^{q-1})} = -\left(w_{2^q}\right)^k$,得到:

\begin{equation}
	\left\{
	\begin{aligned}
		P_{k + 2^{q-1}}^{(q-1)} &= P_k^{(q-1)} \\[8pt]
		I_{k + 2^{q-1}}^{(q-1)} &= I_k^{(q-1)}
	\end{aligned}
	\right.
	\quad k = 0, 1, \ldots, 2^{q-1} -1
\end{equation}

因此有公式:
\begin{equation}
	\left\{
	\begin{aligned}
		y_k &= \frac{1}{2} \left[ P_k^{(q-1)} + \left(w_{2^q}\right)^{-k} I_k^{(q-1)} \right], \quad k = 0, 1, \ldots, 2^{q-1} -1 \\[8pt]
		y_{k + 2^{q-1}} &= \frac{1}{2} \left[ P_k^{(q-1)} - \left(w_{2^q}\right)^{-k} I_k^{(q-1)} \right], \quad k = 0, 1, \ldots, 2^{q-1} -1
	\end{aligned}
	\right.
\end{equation}

特别地:
\begin{equation}
	P_k^{(q-1)} = \text{DFT} \left\{ x_{2j} \right\}_{j=0}^{2^{q-1} -1}, \quad k = 0, 1, \ldots, 2^{q-1} -1
\end{equation}

\begin{equation}
	I_k^{(q-1)} = \text{DFT} \left\{ x_{2j+1} \right\}_{j=0}^{2^{q-1} -1}, \quad k = 0, 1, \ldots, 2^{q-1} -1
\end{equation}

故有:
\begin{equation}
	\left\{
	\begin{aligned}
		P_k^{(q-1)} &= P_k^{(q-2)} + \left(w_{2^{q-1}}\right)^{-k} I_k^{(q-2)} \\[8pt]
		I_k^{(q-1)} &= P_k^{(q-2)} + \left(w_{2^{q-1}}\right)^{-k} I_k^{(q-2)}
	\end{aligned}
	\right.
	\quad k = 0, 1, \ldots, 2^{q-2} -1
\end{equation}

	\subsubsection{FFT 算法中乘法和加法次数的计算}
			$M_q$ :在 FFT 算法中的乘法次数 ,数据长度为 N = $2^q$  \\
			$A_q$ :在 FFT 算法中的加法次数
		

	已知:
	\begin{equation}
			\left\{
		\begin{aligned}
			M_1 &= 0 & A_1 &= 2 \\
			M_2 &= 1 & A_2 &= 8 \\
			M_3 &= 5 & A_3 &= 24
		\end{aligned}
			\right.
	\end{equation}
	
	下面计算 \(M_q\) 和 \(A_q\):
	
	首先讨论 \(M_q\) 和 \(M_{q-1}\) 的关系以及 \(A_q\) 和 \(A_{q-1}\) 的关系:
	

计算 \(P_k^{(q-1)}\) (DFT of even)\quad 乘法次数:\(M_{q-1}\) \quad 加法次数:\(A_{q-1}\)

计算 \(I_k^{(q-1)}\) \quad 乘法次数:\(M_{q-1}\) \quad 加法次数:\(A_{q-1}\)

 乘法次数:\(2^{q-1} - 1\) \quad 加法次数:\(2^q\)

	
	\begin{equation}
		\begin{aligned}
			\left\{
			\begin{aligned}
				M_q &= 2 M_{q-1} + 2^{q-1} - 1 \\
				M_1 &= 0
			\end{aligned}
			\right.
		\end{aligned}
	\end{equation}
	
	\begin{equation}
		\begin{aligned}
			\left\{
			\begin{aligned}
				A_q &= 2 A_{q-1} + 2^q \\
				A_1 &= 2
			\end{aligned}
			\right.
		\end{aligned}
	\end{equation}
	
\begin{table}[h]
	\centering
	\caption{FFT 算法中乘法和加法次数的计算步骤}
	\begin{tabular}{|c|c|c|}
		\hline
		计算步骤 & 乘法次数 & 加法次数 \\
		\hline
		计算 \(P_k^{(q-1)}\) & \(M_{q-1}\) & \(A_{q-1}\) \\
		\hline
		计算 \(I_k^{(q-1)}\) & \(M_{q-1}\) & \(A_{q-1}\) \\
		\hline
		公式 (1) 乘法 & \(2^{q-1} - 1\) & \(2^q\) \\
		\hline
	\end{tabular}
\end{table}
	
	\begin{example}
	\begin{equation}\label{eq:2.3.1}
		\left\{
		\begin{aligned}
			A_q &= 2 A_{q-1} + 2^q  \\[8pt]
			A_1 &= 2
		\end{aligned}
		\right.
	\end{equation}
	\end{example}
	
	\begin{solution}
	由\eqref{eq:2.3.1}式知 \(A_{q-j} = 2 A_{q-j-1} + 2^{q-j}\)
	
	则
	\begin{equation}\label{eq:2.3.2}
	2^j A_{q-j} = 2^{j+1} A_{q-j-1} + 2^q 
	\end{equation}
	
	
	令\eqref{eq:2.3.2}中 \(j = 0,1,\ldots,q-2\),然后相加:
	
\begin{equation}
	\left\{
	\begin{aligned}
		A_q &= 2 A_{q-1} + 2^q \\[8pt]
		2 A_{q-1} &= 2^2 A_{q-2} + 2^q \\[8pt]
		2^2 A_{q-2} &= 2^3 A_{q-3} + 2^q \\[8pt]
		&\vdots \\[8pt]
		2^{q-2} A_2 &= 2^{q-1} A_1 + 2^q
	\end{aligned}
	\right.
\end{equation}
	由上可得
		\begin{equation}
	 A_q = 2^{q-1} A_1 + (q-1) 2^q
		\end{equation}
		则 
		\begin{equation}
		A_q = 2^q + (q-1) 2^q = q 2^q = N \log_2 N
	\end{equation}
		
		因此加法计算量为 \(N \log_2 N = O(N \log N)\)
	
	
	\end{solution}

	
	
		\begin{example}
	
	
	\begin{equation}\label{eq:2.3.21}
		\left\{
		\begin{aligned}
			M_q &= 2 M_{q-1} + 2^{q-1} -1 \\[8pt]
			M_1 &= 0
		\end{aligned}
		\right.
	\end{equation}
	
		\end{example}
	
		\begin{solution}
	由\eqref{eq:2.3.21}式知 
	
	\begin{equation}
	M_{q-j} = 2 M_{q-j-1} + 2^{q -j -1} -1, \quad j=0,1,\ldots,q-2
\end{equation}
	
	则
	\begin{equation}
	 2^j M_{q-j} = 2^{j+1} M_{q-j-1} + 2^{q-1} - 2^j
\end{equation}
	
		\end{solution}
	
\end{document}