\documentclass[12pt,a4paper]{article}
\usepackage[UTF8]{ctex}
\usepackage{geometry}
\usepackage{amsmath,amssymb,amsthm}
\usepackage{mathtools,bm}
\usepackage{empheq}
\usepackage{graphicx}
\usepackage{booktabs}
\usepackage[numbers,sort&compress]{natbib}
\usepackage{caption}
\usepackage{enumitem}
\usepackage{chngcntr}

% ========== 页面布局 ==========
\geometry{left=2.5cm,right=2.5cm,top=2.5cm,bottom=2.5cm}
\setlength{\parskip}{0.5em}
\renewcommand{\baselinestretch}{1.2}

% ========== 数学命令 ==========
\newcommand{\diff}{\mathop{}\!\mathrm{d}}
\newcommand{\R}{\mathbb{R}}
\newcommand{\C}{\mathbb{C}}
\newcommand{\Z}{\mathbb{Z}}
\newcommand{\N}{\mathbb{N}}
\DeclareMathOperator{\supp}{supp}

% ========== 编号系统 ==========
\numberwithin{subsection}{section}   % 子小节按章节编号
\numberwithin{subsubsection}{subsection}
\counterwithin{equation}{subsection} % 公式按子小节编号

% ========== 定理环境 ==========
\theoremstyle{plain}
\newtheorem{theorem}{定理}[section]
\newtheorem{lemma}[theorem]{引理}
\newtheorem{proposition}[theorem]{命题}
\newtheorem{corollary}[theorem]{推论}
\newtheorem{solution}{解}[subsection]  % 解按子小节编号

\theoremstyle{definition}
\newtheorem{definition}[theorem]{定义}
\newtheorem{example}{例题}[subsection]  % 示例按子小节编号

\theoremstyle{remark}
\newtheorem{remark}{注记}[subsection] % 直接在定义时指定计数器的层级关系

\theoremstyle{remark}
\newtheorem{verification}[theorem]{验证}

% 加载hyperref包以实现超链接功能
\usepackage[colorlinks=true, linkcolor=black]{hyperref}

\title{傅里叶变换的应用}
\author{陈柏均}
\date{2025年5月13日}

\begin{document}
	\maketitle

\section{一元傅里叶变换的应用}

\subsection{方程的频率方法(常微分方程)}
\begin{equation}
	\mathcal{F} D^\alpha = (i \omega)^\alpha \mathcal{F}, 
\end{equation}
	当一维时,
\begin{equation}
 (f')^\wedge(\omega) = (i \omega) \hat{f}(\omega) 
\end{equation}

其中,\(D^\alpha = D_1^{\alpha_1} D_2^{\alpha_2} \cdots D_n^{\alpha_n}\),\(D_j = \frac{\partial}{\partial x_j}\),\(X = (x_1, \ldots, x_n)\),\(X^\alpha = x_1^{\alpha_1} \cdots x_n^{\alpha_n}\)

\begin{example}
	
		\begin{equation}\label{1.3}
	\int_0^{+\infty} g(w) \sin w x dw = f(x)
		\end{equation}
	求积分方程\eqref{1.3}的解 \(g(w)\)
	
	其中
	\begin{equation}
		f(x) = 
		\begin{cases} 
			\frac{\pi}{2} \frac{e^{-x}}{x}, & x > 0 \\[8pt]
						0, & x \leq 0
		\end{cases}
	\end{equation}
\end{example}

\begin{solution}
\textbf{方法一}	:用正弦积分变换(略),见教材 P99 例 1.6

\textbf{方法二}
		假设 \(g\) 为奇函数,则:
	\begin{equation}
		\begin{aligned}
			g(w) &= \frac{1}{\sqrt{2 \pi}} \int_{-\infty}^{\infty} \hat{g}(w) e^{-i w x} dw = \frac{1}{\sqrt{2 \pi}} \int_{-\infty}^{\infty} \hat{g}(w) (-i) \sin w x dw \\[8pt]
			&= \frac{-i}{\sqrt{2 \pi}} \int_0^{\infty} g(w) \sin w x dw,
		\end{aligned}
	\end{equation}
		
		由上可得
		\begin{equation}
 \frac{\sqrt{2 \pi}}{-2 i} \hat{g}(x) = f(x)
\end{equation}

 又 \(g\) 为奇函数,有 \(\hat{g}(-w) = -\hat{g}(w)\),故方程变为:
	\begin{equation}\label{1.6}
		\frac{\sqrt{2 \pi}}{-2 i} \hat{g}(t) = f(t)
	\end{equation}
	最后解得 \(g(w)\) 为奇函数,仍记为 \(f\).
\end{solution}

\begin{equation}
	\begin{aligned}
		& f(t)=\frac{1}{\sqrt{2\pi}}\int_{-\infty}^{\infty} f(w)e^{-iwt}dw=\frac{-2i}{\sqrt{2\pi}}\int_0^{+\infty}f(w)\sin w t dw \\[8pt]
		&\qquad=\frac{-2i}{\sqrt{2\pi}}\int_0^\pi \frac{\pi}{2}\sin w t\sin w t dw \\[8pt]
		&\qquad=\frac{-2i}{\sqrt{2\pi}}\cdot\frac{\pi}{4}\int_0^\pi [\cos(1-t)w-\cos(1+t)w] dw \\[8pt]
		&\qquad=\frac{-2i}{\sqrt{2\pi}}\cdot\frac{\pi}{4}\left( \frac{\sin(1-t)\pi}{1-t}+\frac{\sin(1+t)\pi}{1+t} \right) \\[8pt]
		&\qquad=\frac{-2i}{\sqrt{2\pi}}\cdot\frac{\pi}{4}\cdot 2\cdot\frac{\sin\pi t}{1-t^2}=\frac{-2i}{\sqrt{2\pi}}\cdot\frac{\pi}{2}\cdot\frac{\sin \pi t}{1-t^2} \\
	\end{aligned}
\end{equation}

对\eqref{1.6}两边做傅里叶变换
\begin{equation}
	\frac{\sqrt{2\pi}}{-2i} \hat{g}(t) = f(t)
\end{equation}
利用做两次傅里叶变换是镜像对称变换的性质
\begin{equation}\label{2f}
	\mathcal{F}^{2}\{f(t)\} = \mathcal{F}\{\mathcal{F}\{f(t)\}\} = f(-t)
\end{equation}

\begin{equation}
	\frac{\sqrt{2\pi}}{-2i} \hat{g}(-t) = \hat{f}(t)
\end{equation}
而且\(g\)为奇函数可知
\begin{equation}
	g(-t) = \frac{-2i}{\sqrt{2\pi}} \cdot \frac{\pi}{2} \cdot \frac{\sin \pi t}{1 - t^2}
\end{equation}

\begin{equation}
	g(t) = \frac{\sin\pi t}{1 - t^2}
\end{equation}
			
\begin{example}
	积分方程解
	\begin{equation}
	  g(t) = h(t) + \int_{-\infty}^{\infty} f(t) g(t - x) dx
\end{equation}
	\( h \)、\( f \) 已知,且 \( g \)、\( h \)、\( f \) 的傅里叶变换存在。
	
	\begin{solution}
		由傅里叶变换的卷积公式:
		\begin{equation}
			\mathcal{F}\{f * g\} = \sqrt{2\pi} \cdot \hat{f} \cdot \hat{g}
		\end{equation}
		
		原方程可化为:
		\begin{equation}
			\hat{g}(w) = \hat{h}(w) + \hat{f}(w) \cdot \hat{g}(w) \cdot \sqrt{2\pi}
		\end{equation}
		
		解得:
		\begin{equation}
			\hat{g}(w) = \frac{\hat{h}(w)}{1 - \sqrt{2\pi} \cdot \hat{f}(w)}
		\end{equation}
		
		因此:
		\begin{equation}
			g(t) = \frac{1}{\sqrt{2\pi}} \int_{-\infty}^{\infty} \frac{\hat{h}(w)}{1 - \sqrt{2\pi} \cdot \hat{f}(w)} e^{iwt} dw
		\end{equation}
		
	\end{solution}
\end{example}

\begin{example}
	常微分非齐次线性积分方程:
	\begin{equation}
		y'' - y = -f
	\end{equation}
 其中$  f$为已知函数
	 
	\begin{solution}
		对两边取傅里叶变换:
		\begin{equation}
			(iw)^2 \hat{y}(w) - \hat{y}(w) = -\hat{f}(w)
		\end{equation}
		
		解得:
		\begin{equation}
			\hat{y}(w) = \frac{\hat{f}(w)}{1 + w^2}
		\end{equation}
		
		因此:
		\begin{equation}
			y(x) = \frac{1}{\sqrt{2\pi}} \int_{-\infty}^{\infty} \frac{\hat{f}(w)}{1 + w^2} e^{iwx} dw
		\end{equation}
		
		将 \(\frac{\hat{f}(w)}{1 + w^2}\) 视为 \(\hat{f}(w)\) 与 \(\frac{1}{1 + w^2}\) 的乘积。
		
		由卷积定理:
		\begin{equation}
			(f * h)(w) = \sqrt{2\pi} \hat{f}(w) \cdot \hat{h}(w), \quad \text{其中} \, \hat{h}(w) = \frac{1}{1 + w^2}.
		\end{equation}
		
		因此:
		\begin{equation}
			\hat{y}(w) = \sqrt{2\pi} \hat{f}(w) \cdot \hat{g}(w), \quad \text{其中} \, \hat{g}(w) = \frac{1}{\sqrt{2\pi}} \cdot \frac{1}{1 + w^2}.
		\end{equation}
		
		所以:
		\begin{equation}
			y(t) = (f * g)(t), \quad \text{其中} \, g(w) = \frac{1}{2} e^{-|t|}.
		\end{equation}
		
		因此:
		\begin{equation}
			y(t) = \frac{1}{2} \int_{-\infty}^{\infty} f(t) e^{-|t - x|} dx.
		\end{equation}
		
	\end{solution}
\end{example}


\begin{example}
	求解积分方程:
	\begin{equation}
		a x'(t) + b x(t) + c \int_{0}^{t} x(t) dt = h(t),
	\end{equation}

	其中 $ a, b, c \in \mathbb{R}$,  $h  $已知
	
	
	\begin{solution}
		关键是通过傅里叶变换求解。设 \(G' = g\),且 \(G\) 有傅里叶变换。
		
		对两边取傅里叶变换:
		\begin{equation}
			(iw) \hat{G}(w) = \hat{g}(w) \Rightarrow \hat{G}(w) = \frac{\hat{g}(w)}{iw}.
		\end{equation}
		
		利用此公式,原方程可变换为:
		\begin{equation}
			a (iw) \hat{x}(w) + b \hat{x}(w) + c \frac{\hat{x}(w)}{iw} = h(w)
		\end{equation}
		
		解得:
		\begin{equation}
			\hat{x}(w) = \frac{h(w)}{iaw + b + \frac{c}{iw}}.
		\end{equation}
		
	\end{solution}
\end{example}

\subsection{PDE的傅里叶变换}

\subsubsection{一维波动方程初值问题}

\begin{example}
	求解一维波动方程初值问题:
\begin{subequations}
	\begin{empheq}[left=\empheqlbrace]{align}
		\frac{\partial^2 u}{\partial t^2} &= \frac{\partial^2 u}{\partial x^2}, \quad x \in \mathbb{R}, t > 0 \label{eq:pde} \\[8pt]
		u|_{t=0} &= \varphi_0(x) \label{eq:initial_u} \\[8pt]
		\frac{\partial u}{\partial t}\bigg|_{t=0} &= \varphi_1(x) \label{eq:initial_ut}
	\end{empheq}
\end{subequations}
\end{example}

\begin{solution}
	对二元函数 \(u(x, t)\) 的 \(x\) 变量作傅里叶变换,记之为 \(V(w, t)\)。
	
	则:
	\begin{equation}
		V(w, t) = \frac{1}{\sqrt{2\pi}} \int_{-\infty}^{\infty} u(x, t) e^{-iwx} dx
	\end{equation}
	
	于是:
	\begin{equation}
		\frac{\partial V}{\partial t}(w, t) = \frac{1}{\sqrt{2\pi}} \int_{-\infty}^{\infty} \frac{\partial u}{\partial t}(x, t) e^{-iwx} dx = \mathcal{F}_1\left( \frac{\partial u}{\partial t}(x, t) \right)
	\end{equation}
	
	有以下6个公式
\begin{equation}
	\mathcal{F}_1\left( \frac{\partial u}{\partial x} \right)(w, t) = \frac{\partial}{\partial t} V(w, t) \quad \text{(视 \(w\) 为常数)}
\end{equation}

	
\begin{equation}
	\mathcal{F}_1\left( \frac{\partial u}{\partial x} \right)(w, t) = (iw) V(w, t)
\end{equation}

\begin{equation}
	\mathcal{F}_1\left( \frac{\partial^2 u}{\partial x^2} \right)(w, t) = \frac{\partial^2}{\partial t^2} V(w, t)
\end{equation}

\begin{equation}
	\mathcal{F}_1\left( \frac{\partial^2 u}{\partial x^2} \right)(w, t) = (iw)^2 V(w, t)
\end{equation}


\begin{equation}
	\mathcal{F}_1\left( \sin x \right)(w) = \sqrt{\frac{\pi}{2}} i \left[ \delta(w + 1) - \delta(w - 1) \right]
\end{equation}

\begin{equation}
	\mathcal{F}_1\left( \cos x \right)(w) = \sqrt{\frac{\pi}{2}} \left[ \delta(w + 1) + \delta(w - 1) \right]
\end{equation}
	
\end{solution}

\begin{equation}
	\mathcal{F}_1(\cos x)(w) = \sqrt{\frac{\pi}{2}} \left[ \delta(w + 1) + \delta(w - 1) \right]
\end{equation}


原方程在频率域可转化为常微分方程问题
\begin{subequations}
	\begin{empheq}[left=\empheqlbrace]{align}
		\frac{d^2 V}{dt^2} &= -w^2 V \label{eq:sub1} \\[8pt]
		V|_{t=0} &= \sqrt{\frac{\pi}{2}} \left[ \delta(w + 1) + \delta(w - 1) \right] \label{eq:sub2} \\[8pt]
		\left. \frac{dV}{dt} \right|_{t=0} &= \sqrt{\frac{\pi}{2}} \cdot i \left[ \delta(w + 1) - \delta(w - 1) \right] \label{eq:sub3}
	\end{empheq}
\end{subequations}

通解如下:
\begin{equation}
V(w, t) = C_1 \sin(wt) + C_2 \cos(wt)
\end{equation}


\begin{remark}
	齐次方程特征方程为 \(\lambda^2 + w^2 = 0\),其解为 \(\lambda = \pm iw\),对应的解为 \(e^{\pm iwt}\),即 \(\cos(wt)\) 和 \(\sin(wt)\)。
\end{remark}

由\eqref{eq:sub2}
\begin{equation}
	C_2 = \sqrt{\frac{\pi}{2}} \left[ \delta(w + 1) + \delta(w - 1) \right]
\end{equation}

由\eqref{eq:sub3}
\begin{equation}
	C_1 = \sqrt{\frac{\pi}{2}} \cdot \frac{i}{w} \left[ \delta(w + 1) - \delta(w - 1) \right]
\end{equation}

最后得
\begin{equation}
	\begin{aligned}
		V(w, t) &= \sqrt{\frac{\pi}{2}} \cdot \frac{i}{w} \left[ \delta(w + 1) - \delta(w - 1) \right] \sin(wt) \\[8pt]
		&\quad + \sqrt{\frac{\pi}{2}} \left[ \delta(w + 1) + \delta(w - 1) \right] \cos(wt) \\[8pt]
		&= \left( \sqrt{\frac{\pi}{2}} \cdot \frac{i}{w} \sin(wt) + \sqrt{\frac{\pi}{2}} \cos(wt) \right) \delta(w + 1) \\[8pt]
		&\quad + \left( -\sqrt{\frac{\pi}{2}} \cdot \frac{i}{w} \sin(wt) + \sqrt{\frac{\pi}{2}} \cos(wt) \right) \delta(w - 1)
	\end{aligned}
\end{equation}

做傅里叶逆变换,从频率域变回时间域
\begin{equation}
	\begin{aligned}
		u(x, t) &= \mathcal{F}_1^{-1}(V(\cdot, t))(x) \\[8pt]
		&= \frac{1}{\sqrt{2\pi}} \int_{-\infty}^{\infty} \left\{ \left[ \sqrt{\frac{\pi}{2}} \cdot \frac{i}{w} \sin(wt) + \sqrt{\frac{\pi}{2}} \cos(wt) \right] \delta(w + 1) \right. \\[8pt]
		&\quad \left. + \left[ -\sqrt{\frac{\pi}{2}} \cdot \frac{i}{w} \sin(wt) + \sqrt{\frac{\pi}{2}} \cos(wt) \right] \delta(w - 1) \right\} e^{iwx} dw \\[8pt]
		&= \frac{1}{\sqrt{2\pi}} \left\{ \left[ \sqrt{\frac{\pi}{2}} \cdot \frac{i}{-1} \sin(-t) + \sqrt{\frac{\pi}{2}} \cos(-t) \right] e^{-ix} \right. \\[8pt]
		&\quad \left. + \left[ \sqrt{\frac{\pi}{2}} \cdot \frac{i}{1} \sin(t) + \sqrt{\frac{\pi}{2}} \cos(t) \right] e^{ix} \right\} \\[8pt]
		&= \frac{1}{2} \left\{ \left[ -i \sin(t) + \cos(t) \right] e^{-ix} + \left[ i \sin(t) + \cos(t) \right] e^{ix} \right\} \\[8pt]
		&= \frac{1}{2} \left\{ \cos(t) e^{-ix} + \cos(t) e^{ix} + i \sin(t) e^{ix} - i \sin(t) e^{-ix} \right\} \\[8pt]
		&= \frac{1}{2} \left\{ \cos(t) \left( e^{ix} + e^{-ix} \right) + i \sin(t) \left( e^{ix} - e^{-ix} \right) \right\} \\[8pt]
		&= \frac{1}{2} \left\{ 2 \cos(t) \cos(x) + 2i \sin(t) \sin(x) \right\} \\[8pt]
		&= \cos(t - x)
	\end{aligned}
\end{equation}


\subsubsection{一维热传导方程}
\begin{example}
	求解一维热传导方程初值问题:
	\begin{subequations} % Governs numbering for the equations inside empheq
		\begin{empheq}[left=\empheqlbrace]{align} % empheq for the brace, align for the equations themselves
			\dfrac{\partial u}{\partial t} &= a^2 \dfrac{\partial^2 u}{\partial x^2} + f(x,t),  x \in \R, t > 0 \label{eq:heat_pde_line}\\[8pt]
			u|_{t=0} &= \varphi(x) \label{eq:heat_ic_line}
		\end{empheq}
	\end{subequations}
\end{example}

\begin{solution}
	令 \(V(w,t) = \mathcal{F}_1(u(\cdot,t))(w)\).
	则
\begin{subequations}
\begin{empheq}[left=\empheqlbrace]{align}
	\mathcal{F}_1\left(\frac{\partial u}{\partial t}(\cdot,t)\right) &= \frac{\partial}{\partial t}(\mathcal{F}_1 u(w,t)) = \frac{\partial}{\partial t}V(w,t) = \frac{dV}{dt}(w,t) \label{eq:fourier_time}\\[8pt]
	\mathcal{F}_1\left(\frac{\partial^2 u}{\partial x^2}\right)(w) &= (iw)^2 V(w,t) = -w^2 V(w,t) \label{eq:fourier_space}
\end{empheq}
\end{subequations}
	记
\begin{subequations}
	\begin{empheq}[left=\empheqlbrace]{align}
		\hat{f}_1(w,t) &= \mathcal{F}_1 f(\cdot,t)(w) \label{eq:fourier_f} \\[8pt]
		\hat{\varphi}(w) &= (\mathcal{F}\varphi)(w) \label{eq:fourier_phi}
	\end{empheq}
\end{subequations}
	则原方程在频域表示为:
	\begin{subequations}
		\begin{empheq}[left=\empheqlbrace]{align}
			\frac{dV}{dt} &= -a^2 w^2 V + \hat{f}_1(w,t) \label{eq:ode_1} \\[8pt]
			V|_{t=0} &= \hat{\varphi}(w) \label{eq:ode_initial}
		\end{empheq}
	\end{subequations}
\end{solution}
方程\eqref{eq:ode_1}为一阶非齐次常微分方程。由常数变异法,可得解如下:
\begin{equation}
	V(w, t) = \hat{\varphi}(w) e^{-a^2 w^2 t} + \int_0^t \hat{f}_1(w, \tau) e^{-a^2 w^2 (t - \tau)} d\tau 
\end{equation}

③
\begin{equation}
	\mathcal{F}^{-1}\left(e^{-a^2 w^2 t}\right)(x) = \sqrt{2\pi} \cdot \dfrac{1}{2a\sqrt{\pi t}} e^{-\dfrac{x^2}{4a^2 t}}
\end{equation}
其中,
\begin{equation}
	\mathcal{F}\left(e^{-\beta x^2}\right)(w) = \dfrac{\sqrt{\pi}}{\sqrt{\beta}} e^{-\dfrac{w^2}{4\beta}}
\end{equation}
令 $\beta = a^2 t$,则
\begin{equation}
	\mathcal{F}^{-1}\left(e^{-a^2 w^2 t}\right)(x) = \dfrac{1}{a\sqrt{2\pi t}} e^{-\dfrac{x^2}{4a^2 t}}
\end{equation}
进一步地,
\begin{equation}
	\mathcal{F}^{-1}\left(e^{-a^2 (w - k)^2 t}\right)(y) = \dfrac{1}{a\sqrt{2\pi (t - \tau)}} e^{-\dfrac{(y - x)^2}{4a^2 (t - \tau)}}
\end{equation}
最终得解:
\begin{equation}
	\begin{aligned}
		u(x, t) &= \left( \varphi(x) * \dfrac{1}{a\sqrt{2\pi t}} e^{-\dfrac{x^2}{4a^2 t}} \right)(x) \\
		&\quad + \int_0^t \left( f(x, \tau) * \dfrac{1}{a\sqrt{2\pi (t - \tau)}} e^{-\dfrac{x^2}{4a^2 (t - \tau)}} \right)(x) d\tau
	\end{aligned}
\end{equation}

\subsubsection{附}
\begin{example}
	求解微分方程初值问题:
	\begin{subequations} 
		\begin{empheq}[left=\empheqlbrace]{align} 
			\dfrac{dy}{dx} &= A y + f(x),  x \in \R \\[8pt]
			y(0) &= C
		\end{empheq}
	\end{subequations}
\end{example}

\begin{solution}
齐次方程 $y' = A y$。解得:
\begin{equation}
	\frac{y'}{y} = A \implies (\ln y)' = A \implies \ln y = A x + C_1 \implies y = e^{C_1} e^{A x}
\end{equation}
即 $y = C e^{A x}$。

 令 $y = C(x) e^{A x}$。代入原方程:
\begin{equation}
	y' = C'(x) e^{A x} + C(x) e^{A x} A
\end{equation}
代入 $y' = A y + f(x)$:
\begin{equation}
	C'(x) e^{A x} + C(x) e^{A x} A = A C(x) e^{A x} + f(x)
\end{equation}
解得:
\begin{equation}
	C'(x) = f(x) e^{-A x}
\end{equation}
因此:
\begin{equation}
	C(x) = \int f(x) e^{-A x} dx
\end{equation}
特解:
\begin{equation}
	C(x) = \int_0^x f(x) e^{-A x} dx + C
\end{equation}

解为:
\begin{equation}
	y = \int_0^x f(x) e^{-A x} dx \cdot e^{A x} + C e^{A x}
\end{equation}

当 $y(0) = C$ 时,解得 $C = C$。

因此通解:
\begin{equation}
	y(x) = C e^{A x} + \int_0^x f(x) e^{A (x - x)} dx
\end{equation}


\subsubsection{求解上半平面无源静电场内电势的边值问题}
\end{solution}

\begin{example}
	求解上半平面无源静电场内电势的边值问题(拉普拉斯方程):
	\begin{subequations} 
		\begin{empheq}[left=\empheqlbrace]{align} 
			\dfrac{\partial^2 u}{\partial x^2} + \dfrac{\partial^2 u}{\partial y^2} = 0,  x \in \mathbb{R}, y > 0 \label{eq:laplace}\\[8pt]
			u|_{y=0} = f(x) \label{eq:boundary}
		\end{empheq}
	\end{subequations}
	
\begin{solution}
	记 $V(w, y) = \mathcal{F}_1 u(x, y)(w)$。
	
	将原方程化为:
	\begin{subequations}
		\begin{empheq}[left=\empheqlbrace]{align}
			- w^2 V(w, y) + \dfrac{\partial^2}{\partial y^2} V(w, y) &= 0 \label{eq:fourier_laplace}\\[8pt]
			V(w, y)|_{y=0} &= \hat{f}(w) \label{eq:fourier_boundary}\\[8pt]
			\lim_{y \to +\infty} V(w, y) &= 0 \label{eq:fourier_infty}
		\end{empheq}
	\end{subequations}
	
	视 $w$ 为常数,这是一个二阶常微分方程。特征方程为:
	\begin{equation}
		\lambda^2 - w^2 = 0 \implies \lambda = \pm |w|
	\end{equation}
	通解为:
	\begin{equation}
		V(w, y) = C_1 e^{|w| y} + C_2 e^{-|w| y}
	\end{equation}
	
	由边界条件:
	\begin{equation}
		V(w, y)|_{y=0} = \hat{f}(w) \implies C_1 + C_2 = \hat{f}(w)
	\end{equation}
	以及
	\begin{equation}
		\lim_{y \to +\infty} V(w, y) = 0 \implies C_1 = 0
	\end{equation}
	
	因此:
	\begin{equation}
		V(w, y) = \hat{f}(w) e^{-|w| y}
	\end{equation}
	
利用傅里叶逆变换:
\begin{equation}
	\mathcal{F}^{-1}\left(e^{-|w| y}\right)(x) = \sqrt{\dfrac{\pi}{y}} \dfrac{y}{x^2 + y^2}
\end{equation}
上式子可由下式得出
	\begin{equation}
	\mathcal{F}\left(e^{-\frac{1}{r}}\right) \implies \sqrt{\dfrac{\pi}{2}} \dfrac{r}{1 + r^2 \xi^2}
\end{equation}
	最终得解
\begin{equation}
	\begin{aligned}
		u(x, y) &= \frac{1}{\sqrt{2 \pi}}\left(f * \sqrt{\frac{2}{\pi}} \frac{y}{x^{2} + y^{2}}\right)(x) \\
		&= \frac{1}{\pi} \int_{-\infty}^{\infty} \frac{y}{x^{2} + y^{2}} f(x - t) \, dt
	\end{aligned}
\end{equation}
	
	上半平面 Poisson 积分核
	\begin{equation}
			P_y(x) = \dfrac{1}{\pi} \cdot \dfrac{y}{y^2 + x^2}
	\end{equation}
	
	\end{solution}
\end{example}

\section{多元傅里叶变换及应用}

\subsection{多元傅里叶变换定义}
\begin{equation}
	\begin{aligned}
		\mathcal{F}f(\vec{w}) &= \mathcal{F}f(w_1, \dots, w_m) \\
		&= \left(\frac{1}{\sqrt{2\pi}}\right)^m \int\limits_{\mathbb{R}^m} f(x_1, \dots, x_m) e^{-i(w_1x_1 + \cdots + w_mx_m)} dx_1 \cdots dx_m \\
		&= \frac{1}{(2\pi)^{m/2}} \int\limits_{\mathbb{R}^m} f(\vec{x}) e^{-i \vec{w} \cdot \vec{x}} d\vec{x}
	\end{aligned}
\end{equation}

\subsection{逆公式}
\begin{equation}
	f(\vec{x}) = \frac{1}{(2\pi)^{m/2}} \int\limits_{\mathbb{R}^m} \hat{f}(\vec{w}) e^{i \vec{w} \cdot \vec{x}} d\vec{w}
\end{equation}

\subsection{偏傅里叶变换}
\begin{equation}
	\mathcal{F}_j f(x_1, \dots, x_{j-1}, w_j, x_{j+1}, \dots, x_m) = \frac{1}{\sqrt{2\pi}} \int\limits_{-\infty}^{\infty} f(\vec{x}) e^{-i w_j x_j} dx_j
\end{equation}

当 $f(\vec{x}) = \prod_{j=1}^m g_j(x_j)$ 变量可分离时,则
\begin{equation}
	(\mathcal{F}f)(\vec{w}) = \prod_{j=1}^m \mathcal{F}_j g_j(w_j)
\end{equation}

\begin{example}
	求函数 \( f(x) = a e^{-b^2 |x|^2} = a e^{-b^2 (x_1^2 + x_2^2)} \) 的二维傅里叶变换 \( \hat{f}(w_1, w_2) \)。
\end{example}

\begin{solution}
	为了求解 \( \hat{f}(w_1, w_2) \),我们首先应用二维傅里叶变换公式:
	\begin{equation}
		\hat{f}(w_1, w_2) = \frac{1}{(2\pi)^2} \int_{-\infty}^{\infty} \int_{-\infty}^{\infty} a e^{-b^2 (x_1^2 + x_2^2)} e^{-i(w_1 x_1 + w_2 x_2)} dx_1 dx_2
	\end{equation}
	
	我们可以将积分拆分为两个独立的一维积分:
	\begin{equation}
		\hat{f}(w_1, w_2) = \frac{1}{(2\pi)^2} \left( \int_{-\infty}^{\infty} a e^{-b^2 x_1^2} e^{-i w_1 x_1} dx_1 \right) \left( \int_{-\infty}^{\infty} e^{-b^2 x_2^2} e^{-i w_2 x_2} dx_2 \right)
	\end{equation}
	
	利用傅里叶变换的性质,对高斯函数 \( e^{-\beta x^2} \) 的傅里叶变换公式为:
	\begin{equation}
		\mathcal{F}(e^{-\beta x^2})(w) = \sqrt{\frac{\pi}{\beta}} e^{-\frac{w^2}{4\beta}}
	\end{equation}
	
	将 \( \beta = b^2 \) 代入上式,得到每个一维积分的结果:
	\begin{equation}
		\int_{-\infty}^{\infty} e^{-b^2 x_j^2} e^{-i w_j x_j} dx_j = \sqrt{\frac{\pi}{b^2}} e^{-\frac{w_j^2}{4 b^2}} \quad (j = 1, 2)
	\end{equation}
	
	因此,二维傅里叶变换的结果为:
	\begin{equation}
		\hat{f}(w_1, w_2) = \frac{a}{4 b^2} \exp\left(-\frac{w_1^2 + w_2^2}{4 b^2}\right)
	\end{equation}
	
	特别地,当 \( a = 1 \), \( b = 1 \) 时,结果简化为:
	\begin{equation}
		\hat{f}(w_1, w_2) = \frac{1}{4} \exp\left(-\frac{w_1^2 + w_2^2}{4}\right)
	\end{equation}
	为算子的不动点。
\end{solution}

\begin{example}
	求函数 \( f(x) = e^{-x^T A x} \) 的傅里叶变换,其中 \( x = (x_1, \dots, x_m)^T \),且 \( A = B^T B \) 为正定矩阵。
\end{example}

\begin{solution}
	首先,我们将函数 \( f(x) \) 表示为:
	\begin{equation}
		e^{-x^T A x} = e^{-x^T B^T B x} = e^{-(B x)^T (B x)}
	\end{equation}
	
	接下来,计算傅里叶变换:
	\begin{equation}
		\hat{f}(w_1, \dots, w_m) = \frac{1}{(2\pi)^m} \int_{\mathbb{R}^m} e^{-(B x)^T (B x)} e^{-i w^T x} dx
	\end{equation}
	
	令 \( y = B x \),则 \( x = B^{-1} y \),且雅可比行列式 \( |B^{-1}| \)。代入上式得到:
	\begin{equation}
		\hat{f}(w) = \frac{1}{(2\pi)^m} \int_{\mathbb{R}^m} e^{-y^T y} e^{-i w^T B^{-T} y} \frac{1}{|B|} dy
	\end{equation}
	
	利用傅里叶变换的性质:
	\begin{equation}
		\mathcal{F}(e^{-|x|^2})(w) = \frac{1}{(\sqrt{2})^m} e^{-\frac{w^T w}{4}}
	\end{equation}
	
	因此,得到:
	\begin{equation}
		\hat{f}(w) = \frac{1}{(\sqrt{2})^m} \frac{1}{|B|} e^{-\frac{(B^{-1} w)^T (B^{-1} w)}{4}}
	\end{equation}
	
	进一步化简为:
	\begin{equation}
		\hat{f}(w) = \frac{1}{(\sqrt{2})^m} \frac{1}{|B|} e^{-\frac{w^T B^{-T} B^{-1} w}{4}}
	\end{equation}
	
	由于 \( A = B^T B \),因此 \( B^{-T} B^{-1} = A^{-1} \),于是:
	\begin{equation}
		\hat{f}(w) = \frac{1}{(\sqrt{2})^m} \frac{1}{|B|} e^{-\frac{w^T A^{-1} w}{4}}
	\end{equation}
	
	特别地,当 \( B = I \) 时,结果变为:
	\begin{equation}
		\hat{f}(w) = \frac{1}{(\sqrt{2})^m} e^{-\frac{w^T w}{4}}
	\end{equation}
\end{solution}

\subsection{性质}

\subsubsection{线性性}
\begin{equation}
	\mathcal{F}(f(x - \vec{a})) = e^{-i \vec{a} \cdot \vec{w}} (\mathcal{F}f)(\vec{w})
\end{equation}
即
\begin{equation}
	F T_{\vec{a}} = M_{-\vec{a}} F.
\end{equation}

\subsubsection{调制性}
\begin{equation}
	F M_{\vec{b}} = T_{\vec{b}} F.
\end{equation}

\subsubsection{微分性质}
\begin{equation}
	\mathcal{F}\left(\dfrac{\partial}{\partial x_j} f\right) = (i w_j) \cdot \mathcal{F}f
\end{equation}
进一步地:
\begin{equation}
	\dfrac{\partial}{\partial w_j} \mathcal{F}f = \mathcal{F}\left(-i x_j f\right)
\end{equation}
即
\begin{equation}
	\dfrac{\partial}{\partial w_j} \hat{f}(\vec{w}) = \mathcal{F}\left(-i x_j f(x)\right)(\vec{w})
\end{equation}

一般化
令 \(\vec{k} = (k_1, \dots, k_m)\),\(D = \left(\dfrac{\partial}{\partial x_1}, \dots, \dfrac{\partial}{\partial x_m}\right)\),则:
\begin{equation}
	D^{\vec{k}} = \dfrac{\partial^{|\vec{k}|}}{\partial x_1^{k_1} \cdots \partial x_m^{k_m}}
\end{equation}
其中 \(|\vec{k}| = k_1 + \cdots + k_m\)。

于是有:
\begin{equation}
	\mathcal{F}(D^{\vec{k}} f)(\vec{w}) = \left(i (w_1, \dots, w_m)^{\vec{k}}\right) (\mathcal{F}f)(\vec{w})
\end{equation}

当 \(P_m\) 为 \(m\) 元多项式时:
\begin{equation}
	\mathcal{F}(P_m(D) f)(\vec{w}) = P_m(i \vec{w}) (\mathcal{F}f)(\vec{w})
\end{equation}

特别地:
\begin{equation}
	D^{\vec{k}} \mathcal{F}f(\vec{w}) = \mathcal{F}((-i x)^{\vec{k}} f)(\vec{w})
\end{equation}

即:
\begin{equation}
	P_m(D) \mathcal{F}f(\vec{w}) = \mathcal{F}(P_m(-i x) f)(\vec{w})
\end{equation}

\subsubsection{卷积}
\begin{equation}
	\mathcal{F}(f * g)(w) = (\sqrt{2\pi})^m (\mathcal{F}f)(w) (\mathcal{F}g)(w)
\end{equation}

\begin{equation}
	\mathcal{F}(f \cdot g)(w) = (\sqrt{2\pi})^{-m} \left( (\mathcal{F}f) * (\mathcal{F}g) \right)(w)
\end{equation}

\subsubsection{多维情况}
对于多维情况,傅里叶变换具有如下性质:
\begin{equation}
	\mathcal{F}\left( f_1 * f_2 * \cdots * f_n \right)(w) = (\sqrt{2\pi})^{m(n-1)} \prod_{j=1}^n (\mathcal{F}f_j)(w)
\end{equation}

\begin{equation}
	\mathcal{F}(\prod_{j=1}^n (f_j)(w)) = (\sqrt{2\pi})^{-m(n-1)} (\mathcal{F}f_1*\mathcal{F}f_2\cdots * \mathcal{F}f_n)(w)
\end{equation}

\subsubsection{缩放}
\begin{equation}
	\mathcal{F}(f(Ax))(w) = |A|^{-\frac{m}{2}} (\mathcal{F}f)(A^{-T} w)
\end{equation}


\section{多元傅里叶变换的应用}

\subsection{二维热传导方程的初值问题}
\begin{example}
	二维热传导方程的初值问题:
	\begin{subequations} 
		\begin{empheq}[left=\empheqlbrace]{align} 
			\dfrac{\partial u}{\partial t} &= a^2 \left( \dfrac{\partial^2 u}{\partial x^2} + \dfrac{\partial^2 u}{\partial y^2} \right), (x, y) \in \mathbb{R}^2, t > 0 \label{eq:2d_heat_pde}\\[8pt]
			u|_{t=0} &= \varphi(x, y) \label{eq:2d_heat_ic}
		\end{empheq}
	\end{subequations}
	其中 \( (x, y) \in \mathbb{R}^2 \), \( t > 0 \)
\end{example}

\begin{solution}
 对三元函数 \( u(x, y, t) \) 的 \( (x, y) \) 变量施加二维傅里叶变换。
	
	记:
	\begin{equation}
		V(w_1, w_2, t) = \mathcal{F} u(\cdot, \cdot, t)(w_1, w_2)
	\end{equation}
	
	由 \(\frac{\partial}{\partial t}\) 与 \(\mathcal{F}\) 的可交换性:
	\begin{equation}
		\mathcal{F}\left( \frac{\partial u}{\partial t} \right)(w_1, w_2, t) = \frac{\partial}{\partial t} \left( \mathcal{F} u \right)(w_1, w_2, t) = \frac{\partial V}{\partial t}(w_1, w_2, t)
	\end{equation}
	
	又:
	\begin{equation}
		\mathcal{F}\left( \frac{\partial^2 u}{\partial x^2} \right)(w_1, w_2, t) = -w_1^2 V(w_1, w_2, t)
	\end{equation}
	
	和:
	\begin{equation}
		\mathcal{F}\left( \frac{\partial^2 u}{\partial y^2} \right)(w_1, w_2, t) = -w_2^2 V(w_1, w_2, t)
	\end{equation}
	
因此,傅里叶变换后的方程及其边界条件为:
\begin{subequations}
	\begin{empheq}[left=\empheqlbrace]{align}
		\frac{\partial V}{\partial t} &= -a^2 (w_1^2 + w_2^2) V \label{eq:fourier_pde} \\[8pt]
		V(w_1, w_2, 0) &= \mathcal{F} \varphi(w_1, w_2) \label{eq:fourier_ic}
	\end{empheq}
\end{subequations}

常微分方程\eqref{eq:fourier_pde}的通解为
\begin{equation}
	V(w_1, w_2, t) = C e^{-a^2 t(w_1^2 + w_2^2)}
\end{equation}

求常数 \( C \),由初始条件 \eqref{eq:fourier_ic} 可得:
\begin{equation}
	C = \hat{\varphi}(w_1, w_2)
\end{equation}

因此:
\begin{equation}
	V(w_1, w_2, t) = \hat{\varphi}(w_1, w_2) e^{-a^2 t(w_1^2 + w_2^2)}
\end{equation}

利用二维傅里叶逆变换:
\begin{equation}
	u(x, y, t) = \mathcal{F}^{-1} \left[ \hat{\varphi}(w_1, w_2) e^{-a^2 t(w_1^2 + w_2^2)} \right]
\end{equation}

其中:
\begin{equation}
	g(x, y, t) = \mathcal{F}^{-1} \left( e^{-a^2 t(w_1^2 + w_2^2)} \right)
\end{equation}

利用分离变量法:
\begin{equation}
	g(x, y, t) = \left( \frac{1}{2a^2 \pi t} \right) e^{-\frac{x^2 + y^2}{4a^2 t}}
\end{equation}

因此,解为:
\begin{equation}
	(x, y, t) = \frac{1}{(2a^2 \pi t)^2} \iint_{\mathbb{R}^2} \varphi(x - \xi, y - \eta) e^{-\frac{\xi^2 + \eta^2}{4a^2 t}} d\xi d\eta
\end{equation}

\begin{remark}
	向量形式:
	\begin{equation}
		u(\vec{x}, t) = \frac{1}{(2a^2 \pi t)^2} \iint_{\mathbb{R}^2} \varphi(\vec{x} - \vec{\xi}) e^{-\frac{|\vec{\xi}|^2}{4a^2 t}} d\vec{\xi}
	\end{equation}
\end{remark}
\end{solution}

\subsection{n微推广}

\begin{equation}
	\begin{cases}
		\dfrac{\partial u}{\partial t} = a^2 \nabla^2 u, & x_j \in \mathbb{R}, j = 1, \dots, n, t > 0 \\[8pt]
		u(\vec{x}, t)|_{t=0} = \varphi(\vec{x})
	\end{cases}
\end{equation}

解为:
\begin{equation}
	u(\vec{x}, t) = \frac{1}{(2 a \sqrt{\pi t})^n} \int_{\mathbb{R}^n} \varphi(\vec{x} - \vec{\xi}) e^{-\frac{|\vec{\xi}|^2}{4 a^2 t}} d\vec{\xi}
\end{equation}

这里:
\begin{equation}
	\nabla^2 u = \sum_{j=1}^n \frac{\partial^2 u}{\partial x_j^2}, \quad \vec{x} = (x_1, \dots, x_n), \quad d\vec{\xi} = d\xi_1 \cdots d\xi_n
\end{equation}


\begin{example}
特别地,三维热传导方程初值问题:
	\begin{subequations} 
		\begin{empheq}[left=\empheqlbrace]{align} 
				\dfrac{\partial u}{\partial t}&= a^2 \left( \dfrac{\partial^2 u}{\partial x^2} + \dfrac{\partial^2 u}{\partial y^2} + \dfrac{\partial^2 u}{\partial z^2} \right) \\[8pt]
		u|_{t=0} &= e^{-(x^2 + y^2 + z^2)}
		\end{empheq}
	\end{subequations}
\end{example}


\begin{solution}
\begin{equation}
	u(x, y, z, t) = \frac{1}{(2 a \sqrt{\pi t})^3} \iiint_{\mathbb{R}^3} e^{-\frac{(x - \xi)^2 + (y - \eta)^2 + (z - \zeta)^2}{4 a^2 t}} d\xi d\eta d\zeta
\end{equation}

注意到被积函数可分离,有:
\begin{equation}
	u(x, y, z, t) = \frac{1}{(2 a \sqrt{\pi t})^3} g_1(x, t) g_2(y, t) g_3(z, t) 
\end{equation}
$g_1 = g_2 = g_3 = g$
其中:
\begin{equation}
	g(x, t) = \int_{-\infty}^{\infty} e^{-(x - \xi)^2} e^{-\frac{\xi^2}{4 a^2 t}} d\xi
\end{equation}

进一步化简:
\begin{equation}
	g(x, t) = \int_{-\infty}^{\infty} \exp\left[ -\left(1 + \frac{1}{4 a^2 t}\right) \xi^2 + 2 x \xi - x^2 \right] d\xi
\end{equation}

\begin{equation}
	\int_{-\infty}^{\infty} e^{-a x^2 + B x + C} dx \quad (a > 0)
\end{equation}

计算过程:
\begin{equation}
	\int_{-\infty}^{\infty} e^{-a x^2 + B x + C} dx = \int_{-\infty}^{\infty} e^{-a \left(x^2 - \frac{B}{a} x - \frac{C}{a}\right)} dx
\end{equation}

配方:
\begin{equation}
	= \int_{-\infty}^{\infty} e^{-a \left[\left(x - \frac{B}{2a}\right)^2 - \frac{B^2}{4a^2} - \frac{C}{a}\right]} dx
\end{equation}

化简:
\begin{equation}
	= e^{\frac{B^2 + 4aC}{4a}} \int_{-\infty}^{\infty} e^{-a \left(x - \frac{B}{2a}\right)^2} dx
\end{equation}

变量替换 \( t = x - \frac{B}{2a} \):
\begin{equation}
	= e^{\frac{B^2 + 4aC}{4a}} \int_{-\infty}^{\infty} e^{-a t^2} dt
\end{equation}

结果:
\begin{equation}
	= \sqrt{\frac{\pi}{a}} e^{\frac{B^2 + 4aC}{4a}}
\end{equation}

特别地,令 \( a = 1 + 4a^2 t \),\( B = 2x \),\( C = -x^2 \):

代入得到:
\begin{equation}
	\frac{1}{\sqrt{a}} e^{\frac{B^2 + 4aC}{4a}} \sqrt{\pi} = \sqrt{\frac{\pi}{1 + 4a^2 t}} e^{\frac{(2x)^2 + 4(1 + 4a^2 t)(-x^2)}{4(1 + 4a^2 t)}}
\end{equation}

化简:
\begin{equation}
	= \sqrt{\frac{\pi}{1 + 4a^2 t}} e^{-\frac{x^2}{1 + 4a^2 t}}
\end{equation}

\begin{equation}
	g(x, t) = \frac{2 a \sqrt{\pi t}}{\sqrt{1 + 4 a^2 t}} \exp\left( -\frac{x^2}{1 + 4 a^2 t} \right)
\end{equation}

因此:
\begin{equation}
	u(x, y, z, t) = \frac{1}{(2 a \sqrt{\pi t})^3} g(x, t) g(y, t) g(z, t)
\end{equation}

\end{solution}




\begin{remark}
	使用傅里叶变换方法求解 PDE 的难点通常在于最后的逆变换步骤。
\end{remark}



\end{document}