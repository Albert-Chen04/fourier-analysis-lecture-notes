\documentclass[12pt,a4paper]{article}
\usepackage[UTF8]{ctex}                    % 中文支持
\usepackage{geometry}                      % 页面布局
\usepackage{amsmath,amssymb,amsthm}        % 数学符号与定理
\usepackage{mathtools,bm}                  % 高级数学工具
\usepackage{graphicx}                      % 图片插入
\usepackage{hyperref}                      % 超链接
\usepackage{booktabs}                      % 三线表
\usepackage[numbers,sort&compress]{natbib} % 参考文献
\usepackage{caption}                       % 图表标题
\usepackage[shortlabels]{enumitem}         % 列表环境

% ========== 页面布局 ==========
\geometry{left=2.5cm, right=2.5cm, top=2.5cm, bottom=2.5cm}
\setlength{\parskip}{0.5em}                % 段落间距
\renewcommand{\baselinestretch}{1.2}       % 行距

% ========== 数学命令 ==========
\newcommand{\diff}{\mathop{}\!\mathrm{d}}  % 微分符号
\newcommand{\R}{\mathbb{R}}                % 实数集
\newcommand{\C}{\mathbb{C}}                % 复数集
\newcommand{\Z}{\mathbb{Z}}                % 整数集
\newcommand{\Lspace}{L^2(-l,l)}            % L²空间

% ========== 定理环境 ==========
\theoremstyle{plain}
\newtheorem{theorem}{定理}[section]
\newtheorem{lemma}[theorem]{引理}
\newtheorem{proposition}[theorem]{命题}

\theoremstyle{definition}
\newtheorem{definition}[theorem]{定义}
\newtheorem{example}[theorem]{示例}
\newtheorem{exercise}{练习}[section]  % 新增练习环境

\theoremstyle{remark}
\newtheorem{remark}[theorem]{注记}

% ========== 文档信息 ==========
\title{初探傅里叶变换}
\author{陈柏均}
\date{\today}

% ========== 正文内容 ==========
\begin{document}
	
	\maketitle
	
	\section{\emph{Fourier级数}到\emph{Fourier变换}的推导}
	傅里叶级数可以解决周期函数和非周期函数但定义域非全实数范围,那如果是定义在 \(\R\) 上的非周期函数我们又该如何解决?
	
	
	假设 \( f \) 是定义在 \(\R\) 上的非周期函数。为得到\emph{Fourier变换}的具体形式,考虑 \( f \) 在区间 \([-l, l]\) 上的\emph{Fourier级数}展开:
	
	\[
	f(x) = \sum_{n=-\infty}^{\infty} \alpha_n e^{in\pi x/l},
	\]
	
	其中系数
	
	\[
	\alpha_n = \frac{1}{2l} \int_{-l}^{l} f(y)e^{-in\pi y/l} dy.
	\]
	
	将系数代入展开式得:
	
	\[
	f(x) = \sum_{n=-\infty}^{\infty} \frac{1}{2l} \int_{-l}^{l} f(y)e^{in\pi (y-x)/l} dy
	\]
	\[
	= \sum_{n=-\infty}^{\infty} \frac{1}{2\pi} \int_{-l}^{l} f(y)e^{in\pi (y-x)/l} dy  \frac{\pi}{l}, \quad -l < y < l.
	\]
	
	\subsection{极限过程推导}
	令 \( t_n = \frac{n\pi}{l} \),\(\Delta t_n = t_{n+1} - t_n = \frac{\pi}{l}\),并定义
	
	\[
	F_l(t_n) = \frac{1}{2\pi} \int_{-l}^{l} f(y)e^{it_n(y-x)} dy,
	\]
	
	则展开式可表示为:
	
	\[
	f(x) = \sum_{n=-\infty}^{\infty} F_l(t_n) \Delta t_n, \quad -l < y < l.
	\]
	当 \( l \to \infty \) 时,上述求和为定积分:
	
	\[
	f(x) = \lim_{l \to +\infty} \sum_{n=-\infty}^{+\infty} F_l(t_n) \Delta t_n = \int_{-\infty}^{+\infty} F_l(t) dt, \quad x \in \R.
	\]
	其中
	
	\[
	F_l(t) = \frac{1}{2\pi} \int_{-\infty}^{+\infty} f(y)e^{it(y-x)} dy
	\]
	最终得
	
	\[
	f(x) = \int_{-\infty}^{+\infty} \left[ \frac{1}{2\pi} \int_{-\infty}^{+\infty} f(y)e^{it(y-x)} dy \right] dt 
	\]
	
	\[
	= \frac{1}{\sqrt{2\pi}} \int_{-\infty}^{\infty} \left( \frac{1}{\sqrt{2\pi}} \int_{-\infty}^{\infty} f(y) e^{-i t y} dy \right) e^{i t x} dt
	\]
	其中
	
	\[
	\hat{f}(t) = \frac{1}{\sqrt{2\pi}} \int_{-\infty}^{\infty} f(y) e^{-i t y} dy
	\]
	\(\hat{f}(t)\)或 $\mathcal{F}[f]$ 称为 \( f \) 的\emph{Fourier变换}
	
	
	\[
	f(x) = \frac{1}{\sqrt{2\pi}} \int_{-\infty}^{\infty} \hat{f}(t) e^{i t x} dt
	\]
	$\mathcal{F}^{-1}[f]$称为\emph{Fourier逆变换}。
	
	\section{傅里叶变换的泛函与拓扑性质}
	
	\subsection{傅里叶变换与逆变换是线性算子}
	
	傅里叶变换 $\mathcal{F}[f]$ 和逆变换 $\mathcal{F}^{-1}[f]$ 都是线性算子。
	
	\textbf{加法性}:
	\[
	\mathcal{F}[f+g] = \mathcal{F}[f] + \mathcal{F}[g]
	\]
	证明:
	\[
	\mathcal{F}[f+g] = \frac{1}{\sqrt{2\pi}} \int_{-\infty}^{+\infty} \big(f(t) + g(t)\big) e^{-i\lambda t} dt
	\]
	由积分的线性性可得:
	\[
	= \frac{1}{\sqrt{2\pi}} \int_{-\infty}^{+\infty} f(t) e^{-i\lambda t} dt + \frac{1}{\sqrt{2\pi}} \int_{-\infty}^{+\infty} g(t) e^{-i\lambda t} dt = \mathcal{F}[f] + \mathcal{F}[g]
	\]
	
	\textbf{齐次性}(数乘不变性):
	\[
	\mathcal{F}[cf] = c\mathcal{F}[f], \quad \text{其中} c \in \mathbb{C}
	\]
	证明:
	\[
	\mathcal{F}[cf] = \frac{1}{\sqrt{2\pi}} \int_{-\infty}^{+\infty} cf(t) e^{-i\lambda t} dt = c \cdot \frac{1}{\sqrt{2\pi}} \int_{-\infty}^{+\infty} f(t) e^{-i\lambda t} dt = c\mathcal{F}[f]
	\]
	
	\textbf{逆变换的线性性}:
	类似地,傅里叶逆变换 $\mathcal{F}^{-1}[f]$ 也满足:
	\[
	\mathcal{F}^{-1}[f+g] = \mathcal{F}^{-1}[f] + \mathcal{F}^{-1}[g], \quad \mathcal{F}^{-1}[cf] = c\mathcal{F}^{-1}[f]
	\]
	证明方法与傅里叶变换完全相同,只需将积分核中的 $e^{-i\lambda t}$ 替换为 $e^{i\lambda t}$。
	
	
	\subsection{傅里叶变换是有界线性算子}
	
	傅里叶变换 $\mathcal{F}[f]$ 是有界线性算子。
	
	对于任意 $f \in L^1(\mathbb{R})$,有:
	\[
	|\mathcal{F}[f](\lambda)| \leq \frac{1}{\sqrt{2\pi}} \|f\|_1
	\]
	证明:
	\[
	\left| \frac{1}{\sqrt{2\pi}} \int_{-\infty}^{+\infty} f(t) e^{-i\lambda t} dt \right|\leq \frac{1}{\sqrt{2\pi}} \int_{-\infty}^{+\infty} |f(t)|\cdot|e^{-i\lambda t}| dt = \frac{1}{\sqrt{2\pi}} \int_{-\infty}^{+\infty} |f(t)| dt = \frac{1}{\sqrt{2\pi}} \|f\|_1
	\]
	
	因此,若 $f \in L^1(\mathbb{R})$,则 $\mathcal{F}[f] \in L^\infty(\mathbb{R})$,且:
	\[
	\|\mathcal{F}[f]\|_{L^\infty} = \operatorname{ess\,sup} |\mathcal{F}[f](\lambda)| \leq \frac{1}{\sqrt{2\pi}} \|f\|_1 < +\infty
	\]
	这表明傅里叶变换将 $L^1$ 空间中的函数映射到 $L^\infty$ 空间中的函数,且变换后的函数有界。
	
\subsection{傅里叶变换\(\mathcal{F}[f](\lambda)\)的连续性}


\[ |F[f](\lambda + \Delta \lambda) - F[f](\lambda)| \]

\[ = \left| \frac{1}{\sqrt{2\pi}} \int_{-\infty}^{+\infty} f(t) \cdot e^{-it(\lambda + \Delta \lambda)} dt - \frac{1}{\sqrt{2\pi}} \int_{-\infty}^{+\infty} f(t) \cdot e^{-it\lambda} dt \right| \]

\[ = \left| \frac{1}{\sqrt{2\pi}} \int_{-\infty}^{+\infty} f(t) \cdot e^{-it\lambda} \cdot (e^{-it\Delta \lambda} - 1) dt \right| \leq \frac{1}{\sqrt{2\pi}} \int_{-\infty}^{+\infty} |f(t)| \cdot |e^{-it\lambda}| \cdot |e^{-it\Delta \lambda} - 1| dt \]

因为

\[ |e^{-it\lambda}| = 1,	|e^{-it\Delta \lambda} - 1| \leq 2 \]

所以

\[ \leq \frac{1}{\sqrt{2\pi}} \int_{-\infty}^{+\infty} |f(t)| \cdot 2 dt = \frac{2}{\sqrt{2\pi}} \int_{-\infty}^{+\infty} |f(t)| dt < +\infty \]

应用勒贝格控制收敛定理,交换极限与积分:

\[ \lim_{\Delta \lambda \to 0} \int_{-\infty}^{+\infty} |f(t)| \cdot |e^{-it\Delta \lambda} - 1| dt = \int_{-\infty}^{+\infty} |f(t)| \cdot \lim_{\Delta \lambda \to 0} |e^{-it\Delta \lambda} - 1| dt \]

\begin{proof}	$ \lim_{\Delta \lambda \to 0} |e^{-it\Delta \lambda} - 1| = 0 $

	
	\[ e^{-it\Delta \lambda} = \cos(t \Delta \lambda) - i \sin(t \Delta \lambda) \]
	
	因此,\(e^{-it\Delta \lambda} - 1\) 可以表示为:
	\[ e^{-it\Delta \lambda} - 1 = \cos(t \Delta \lambda) - 1 - i \sin(t \Delta \lambda) \]
	
	\[ |e^{-it\Delta \lambda} - 1| = \sqrt{(\cos(t \Delta \lambda) - 1)^2 + (\sin(t \Delta \lambda))^2} \]
	
	展开平方项:
	\[ (\cos(t \Delta \lambda) - 1)^2 + (\sin(t \Delta \lambda))^2 = \cos^2(t \Delta \lambda) - 2 \cos(t \Delta \lambda) + 1 + \sin^2(t \Delta \lambda) \]
	
	\[ = 1 - 2 \cos(t \Delta \lambda) + 1 = 2(1 - \cos(t \Delta \lambda)) \]
	
	因此:
	\[ |e^{-it\Delta \lambda} - 1| = \sqrt{2(1 - \cos(t \Delta \lambda))} \]
	
	利用三角恒等式 \(1 - \cos x = 2 \sin^2\left(\frac{x}{2}\right)\):
	\[ |e^{-it\Delta \lambda} - 1| = \sqrt{2 \cdot 2 \sin^2\left(\frac{t \Delta \lambda}{2}\right)} = 2 \left|\sin\left(\frac{t \Delta \lambda}{2}\right)\right| \]
	
	当 \(\Delta \lambda \to 0\) 时,由洛必达法则可得
	\[|e^{-it\Delta \lambda} - 1|= \left|2\sin\left(\frac{t \Delta \lambda}{2}\right)\right| = \left|\frac{2t \Delta \lambda}{2}\right|= |t \Delta \lambda| \]
	
	
	对于$\forall t\in \mathbb{R}$,当 \(\Delta \lambda \to 0\) 时,\(|t \Delta \lambda| \to 0\)
	\[ \lim_{\Delta \lambda \to 0} |e^{-it\Delta \lambda} - 1| = 0 \]
	
\end{proof}

最终得
\[ \lim_{\Delta \lambda \to 0} |F[f](\lambda + \Delta \lambda) - F[f](\lambda)| = 0 \]

这表明傅里叶变换 \(\mathcal{F}[f](\lambda)\) 是连续的。

综上所述,傅里叶变换 \(\mathcal{F}\) 将 \( L^1(\mathbb{R}) \) 中的函数映射到有界连续函数空间 \( C_b(\mathbb{R}) \)
	
	\section{傅里叶变换的性质(续)}
	
	\subsection{平移性 (Translation Property)}
	\[
	\mathcal{F}\big[f(x-a)\big](\lambda) = e^{-i\lambda a} \cdot \mathcal{F}[f](\lambda)
	\]
	\begin{proof}
		
		\[
		\mathcal{F}\big[f(x-a)\big](\lambda) = \frac{1}{\sqrt{2\pi}} \int_{-\infty}^{+\infty} f(x-a) e^{-i\lambda x} dx
		\]
		
		令 $t = x - a$,则 $x = t + a$,$dx = dt$,代入得:
		\[
		= \frac{1}{\sqrt{2\pi}} \int_{-\infty}^{+\infty} f(t) e^{-i\lambda (t+a)} dt
		\]
		
		\[
		= e^{-i\lambda a} \cdot \frac{1}{\sqrt{2\pi}} \int_{-\infty}^{+\infty} f(t) e^{-i\lambda t} dt = e^{-i\lambda a} \cdot \mathcal{F}[f](\lambda)
		\]
	\end{proof}
	
	\subsection{调制性 (Modulation Property)}
	
	\[
	\mathcal{F}\left[e^{iax}f(x)\right](\lambda) = \mathcal{F}[f](\lambda - a)
	\]
	\begin{proof}
		
		\[
		\mathcal{F}\left[e^{iax}f(x)\right](\lambda) = \frac{1}{\sqrt{2\pi}} \int_{-\infty}^{+\infty} e^{iax}f(x) e^{-i\lambda x} dx
		\]
		
		\[
		= \frac{1}{\sqrt{2\pi}} \int_{-\infty}^{+\infty} f(x) e^{-i(\lambda - a)x} dx = \mathcal{F}[f](\lambda - a)
		\]
	\end{proof}
	
	\subsection{伸缩性 (Scaling Property)}
	
	\[
	\mathcal{F}\big[f(bx)\big](\lambda) = \frac{1}{|b|} \mathcal{F}[f]\left( \frac{\lambda}{b} \right)
	\]
	\begin{proof}
		
		\[
		\mathcal{F}\big[f(bx)\big](\lambda) = \frac{1}{\sqrt{2\pi}} \int_{-\infty}^{+\infty} f(bx) e^{-i\lambda x} dx
		\]
		
		令 $u = bx$,则 $x = u/b$,$dx = du/|b|$(这里是因为变量替换,积分上下限会根据单调性改变,$du$也会因为单调性改变,最后都是正的,所以加了绝对值),代入得:
		\[
		= \frac{1}{\sqrt{2\pi}} \int_{-\infty}^{+\infty} f(u) e^{-i\lambda \cdot \frac{u}{b}} \cdot \frac{du}{|b|}
		\]
		
		\[
		= \frac{1}{|b|} \cdot \frac{1}{\sqrt{2\pi}} \int_{-\infty}^{+\infty} f(u) e^{-i\frac{\lambda}{b}u} du = \frac{1}{|b|} \mathcal{F}[f]\left( \frac{\lambda}{b} \right)
		\]
	\end{proof}
	
	\subsection{傅里叶变换综合练习,好像说是会考}
	
	\begin{exercise}
		计算 $\mathcal{F}\left[ f\left( \frac{x-b}{a} \right) \right](w)$
		
		
		\[
		\mathcal{F}\left[ f\left( \frac{x-b}{a} \right) \right](w) = \frac{1}{\sqrt{2\pi}} \int_{-\infty}^{+\infty} f\left( \frac{x-b}{a} \right) e^{-iwx} dx
		\]
		
		令 $u = \frac{x-b}{a}$,则$x = au + b, \quad dx = |a|du$
		
		\[
		= \frac{1}{\sqrt{2\pi}} \int_{-\infty}^{+\infty} f(u) e^{-iw(au+b)} \cdot |a|du
		\]
		
		\[
		= \frac{|a|}{\sqrt{2\pi}} e^{-ibw} \int_{-\infty}^{+\infty} f(u) e^{-i(a w)u} du
		\]
		
		\[
		= |a| e^{-ibw} \cdot \mathcal{F}[f](aw)
		\]
		
	\end{exercise}
	
	\begin{exercise}
		计算 $\mathcal{F}\left[ f(ax-b) \right](w)$
		
		\[
		\mathcal{F}\left[ f(ax-b) \right](w) = \frac{1}{\sqrt{2\pi}} \int_{-\infty}^{+\infty} f(ax-b) e^{-iwx} dx
		\]
		
		令 $u = ax - b$,则$x = \frac{u + b}{a}, \quad dx = \frac{1}{|a|}du$
		
		\[
		= \frac{1}{\sqrt{2\pi}} \int_{-\infty}^{+\infty} f(u) e^{-iw\left(\frac{u+b}{a}\right)} \cdot \frac{1}{|a|}du
		\]
		
		\[
		= \frac{1}{|a|\sqrt{2\pi}} e^{-i\frac{bw}{a}} \int_{-\infty}^{+\infty} f(u) e^{-i\frac{w}{a}u} du
		\]
		
		\[
		= \frac{1}{|a|} e^{-i\frac{bw}{a}} \cdot \mathcal{F}[f]\left( \frac{w}{a} \right)
		\]
	\end{exercise}
	
	\section{傅里叶变换的衰减性,连续的Riemann-Lebesgue定理}
	
	在傅里叶级数的时候我们探讨的整数上的Riemann-Lebesgue定理,得出系数具有一定的衰减性。现在我们考虑连续下的Riemann-Lebesgue定理,得傅里叶变换具有一定的衰减性。
	
	\begin{theorem}[连续的Riemann-Lebesgue定理]
		若 \( f \in L^1(\mathbb{R}) \),则
		\[
		\lim_{|w| \to \infty} \hat{f}(w) = 0
		\]
		即 \( \hat{f}(w) \to 0 \) 当 \( |w| \to \infty \)。
	\end{theorem}
	
	
	
	\begin{proof}
		(1) 当 \( f = \chi_{[a, b]} \) 时,直接计算其\emph{Fourier变换}:
		\[
		\hat{f}(w) = \frac{1}{\sqrt{2\pi}} \int_{a}^{b} e^{-i w t} dt
		\]
		计算积分:
		\[
		\hat{f}(w) = \frac{1}{\sqrt{2\pi}} \cdot \frac{e^{-i w b} - e^{-i w a}}{-i w}
		\]
		整理得:
		\[
		\hat{f}(w) = \frac{1}{\sqrt{2\pi}} \cdot \frac{e^{-i w b} - e^{-i w a}}{-i w}
		\]
		取绝对值:
		\[
		|\hat{f}(w)| \leq \frac{1}{\sqrt{2\pi}} \cdot \frac{2}{|w|} \to 0 \quad (|w| \to \infty)
		\]
		因此,\( \hat{f}(w) \to 0 \) 当 \( |w| \to \infty \)。
		
		(2) 当 \( f \) 是有限个特征函数 \( \chi_{[a_i, b_i]} \) 的线性组合时,利用积分线性性和极限线性性,可知:
		\[
		\hat{f}(w) \to 0 \quad (|w| \to \infty)
		\]
		即紧支撑的分片常函数的Fourier变换趋于0(当 \( |w| \to \infty \) 时)。
		
		(3) 对于任意 \( f \in L^1(\mathbb{R}) \),对于任意 \( \varepsilon > 0 \),存在一个紧支撑的分片常函数 \( g \),使得:
		\[
		\|f - g\|_1 = \int_{\mathbb{R}} |f(x) - g(x)| dx < \varepsilon
		\]
		
		(4) 对于 \( f \in L^1(\mathbb{R}) \),分解其\emph{Fourier变换}:
		\[
		\hat{f}(w) = \hat{f}(w) - \hat{g}(w) + \hat{g}(w)
		\]
		即:
		\[
		\hat{f}(w) = \frac{1}{\sqrt{2\pi}} \int_{\mathbb{R}} (f(x) - g(x)) e^{-i w x} dx + \frac{1}{\sqrt{2\pi}} \int_{\mathbb{R}} g(x) e^{-i w x} dx
		\]
		记为:
		\[
		\hat{f}(w) = J_1(w) + J_2(w)
		\]
		
		估计 \( J_1(w) \) 的绝对值:
		
		\[
		|J_1(w)| \leq \frac{1}{\sqrt{2\pi}} \int_{\mathbb{R}} |f(x) - g(x)| dx \leq \frac{1}{\sqrt{2\pi}} \varepsilon
		\]
		这里利用了 \( L^1 \) 范数的性质,即:
		\[
		\int_{\mathbb{R}} |f(x) - g(x)| dx < \varepsilon
		\]
		
		估计 \( J_2(w) \) 的绝对值:
		
		由于 \( g \) 是紧支撑的分片常函数,根据第(2)部分的结论:
		\[
		|J_2(w)| = o(1) \quad (|w| \to \infty)
		\]
		
		综上所述
		
		将 \( J_1(w) \) 和 \( J_2(w) \) 的估计结果结合起来:
		\[
		|\hat{f}(w)| \leq |J_1(w)| + |J_2(w)| \leq \frac{1}{\sqrt{2\pi}} \varepsilon + o(1)
		\]
		
		由于 \( \varepsilon \) 是任意的,当 \( |w| \to \infty \) 时,\( o(1) \) 趋于0,因此:
		\[
		\hat{f}(w) \to 0 \quad (|w| \to \infty)
		\]
		
	\end{proof}
	
	
	
\end{document}